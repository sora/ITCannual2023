\subsection{高速3次元形状計測技術の評価と応用(宮下令央)}

%  本年度は昨年度開発した高速3次元形状計測技術の評価、および関連する高速画像処理システムの研究・発表を行った。
%  新たに開発した高速3次元形状計測技術によって、従来手法よりも高解像度かつ高精度な3次元形状計測が1,000fpsを超える高速性と低遅延性を維持したまま実現できることを確認し、成果を国内外で発表した。

% また、高速3次元形状計測の関連研究として、従来のテレセントリック光学系による画像検査では困難な長尺の棒材の寸法検測を非テレセントリック光学系によって実現するシステムや、高速3次元形状計測と高速法線計測を統合して高密度かつ高精度かつ高速な3次元形状計測を実現するシステム、さらにイメージセンサに高速画像処理を行うプロセッサを並置したビジョンチップを用いて小型化を実現した高速光軸制御システムについて、論文誌において発表を行った。

% さらに、人間の錯覚を利用して仮想的な運動を付加するダイナミックプロジェクションマッピング技術をコンピュータグラフィックスに応用し、アニメーションのレンダリングを高速化する研究を行い、被験者実験を通して得た人間の知覚特性に関する知見について国際学会において発表を行った。
% 今後は高速3次元形状計測技術を発展させ、質感計測やダイナミックプロジェクションマッピングへの応用を進めていく予定である。

% また、解説論文の執筆、研究会の運営や査読を通して学会への貢献を続けている。
