\subsection{高速知能システムの研究(石川正俊)}

%  本研究室では、センサ情報処理における並列処理を基盤として、高速・リアルタイム性を有するセンサ情報処理を高度に実現し、高速知能システムとして実装する研究を行っている。具体的には、以下4つの分野での研究開発を行っている。また、各分野での新規産業開拓にも力を注ぎ、研究成果の技術移転、共同研究、事業化等を様々な形で積極的に推進している。

% センサフュージョンの研究では、高速のセンサフィードバックに関する理論並びにシステムアーキテクチャの構築、その高速知能ロボットとしての実装、並びに高速性を活かした新規タスクの実現、特に、センサ情報に基づく人間機械協調システムの開発を行っている。

% ダイナミックビジョンシステムの研究では、高速ビジョンや動的光学系に基づき運動対象の情報を適応的に取得する基礎技術の開発、特に、高速光軸制御や適応光学系の技術開発やトラッキング撮像に関する応用システムの開発を行っている。

% システムビジョンデザインの研究では、高速三次元形状計測や高速質感計測など、並列処理に基づく高速画像処理技術 (理論、アルゴリズム、デバイス) の開発とその応用システムの実現を目指し、特に高速画像処理システムの開発や応用システムの開発を行っている。

% アクティブパーセプションの研究では、実世界における新たな知覚補助技術並びにそれに基づく新しい対話の形の創出を目指し、特に各種高速化技術を用いた能動計測や能動認識を利用した革新的情報環境・ヒューマンインタフェイスの開発を行っている。

% これらの研究では、原則的に並列かつリアルタイムの演算構造を有する現実の物理世界と同等の構造を工学的に実現することを目指しており、そのことにより、現実世界の理解を促すばかりでなく、従来のシステムをはるかに凌駕する性能を有する高速知能システムを生み出すことができ、結果として、まったく新しい情報システムを構築することが可能となる。

