\subsection{計算機を介した人と生態系のインタラクションの研究(小林 博樹)}

本研究室は計算機を介した人と生態系のインタラクションの研究の行っている。これまで人間を対象とした知能情報学の見地を、多様で複雑な実世界の生物・環境・地理学・獣医学領域へ応用・発展させる研究である。研究内容はコンピュータ科学、環境学、メディアアート、など多岐に亘っており、特に、計算機を介した人と生態系のインタラクションHCBI(Human-Computer-Biosphere Interaction)の概念を情報学分野で発表し、このテーマを中心に、環境問題の解決を目的として、国内外で研究活動を独自に行ってきた。古典的なコンピュータ科学では、HCI(Human-Computer Interaction)が主要な研究領域の1つとなるが、本研究室はこの研究領域を地球環境にまで拡大すべく、人間と生態系の調和あるインタラクションを実現するシステムを提案し「時空間スケールの大きい環境問題を自律的に解決する情報基盤技術」として、そのフィールドでの実証実験を試みている。つまり、コンピュータ科学の分野では人間が活動する地理空間を対象とした研究が中心であったが、本研究室は人間が活動していない、情報通信技術の応用が困難な地理空間を対象にした情報デザインと野生動物IoTの研究を行っている。このように本研究室は、情報工学をベースとして、特に計算機を用いて生態系と人間のインタラクションを専門として実績をあげている。2022年度から科学技術振興機構の創発的研究支援事業として業務を実施している。
 