\documentclass[11pt]{jarticle}
\usepackage{ITCannual}
\usepackage{amsmath}
\usepackage{amssymb}
\usepackage{times}
\usepackage[dvipdfmx]{graphicx}

%\usepackage[style=numeric]{biblatex}

\title{データ科学研究部門 研究報告}
\author{小林博樹, 鈴村豊太郎, 空閑洋平, Parajuli Laxmi Kumar, 河村光晶, 川瀬純也, 華井雅俊,\\
\textbf{Li Zihui, 上坂怜生, 石川正俊, 早川智彦, 黄守仁, 末石智大, 宮下令央, 田畑智志}}

\begin{document}
\maketitle

\section{データ科学研究部門 概要(昨年の文言)}
データ科学研究部門では、2022度、教授4名(特任教授1名)、准教授3名(特任准教授2名)、講師4名(特任講師4名)、助教5名(特任助教3名)、研究員1名(特任研究員1名)が在籍した。同部門のメンバーは専任教員と特任教員の2つのグループから成る。専任教員はそれぞれが独立して研究活動を行うグループで、特任教員は石川特任教授を中心とする石川グループ研究室である。

% \subsection{専任教員グループの研究テーマ}
%
% \begin{quote}
% \begin{itemize}
% \item 計算機を介した人と生態系のインタラクションの研究(小林)
% \item 大規模グラフニューラルネットワークと様々な実社会問題への応用(鈴村)
% \item データセンタハードウェアへのソフトウェア脆弱試験の適応(空閑)
% \item Title (Kumar)
% \item Title (河村)
% \item データ駆動型知能に基づくアーバンコンピューティング(姜)
% \item 野生動物ワイヤレスセンサネットワーク実証実験基盤構築に向けた研究(川瀬)
% \item グラフニューラルネットワークとその物性予測問題への応用に関する研究(華井)
% \item Title (Zihui)
% \end{itemize}
% \end{quote}

%\subsection{石川研究室全体の研究活動概要}
%\subsection{研究報告(石川グループ研究室)}

%  センサやロボットはもちろんのこと、社会・心理現象等も含めて、現実の物理世界は、原則的に並列かつリアルタイムの演算構造を有している。その構造と同等の構造を工学的に実現することは、現実世界の理解を促すばかりでなく、応用上の様々な利点をもたらし、従来のシステムをはるかに凌駕する性能を生み出すことができ、結果として、まったく新しい情報システムを構築することが可能となる。本研究室では、特にセンサ情報処理における並列処理と高速・リアルタイム性を高度に示現する研究として、以下4つの分野での研究を行っている。また、新規産業分野開拓にも力を注ぎ、研究成果の技術移転,共同研究,事業化等を様々な形で積極的に推進している。

% 五感の工学的再構成を目指したセンサフュージョンの研究では、理論並びにシステムアーキテクチャの構築とその高速知能ロボットの開発、その応用としての新規タスクの実現、特に、視・触覚センサによるセンサ情報に基づく人間機械協調システムの開発を行っている。

% ダイナミックビジョンシステムの研究では、高速ビジョンや動的光学系に基づき運動対象の情報を適応的に取得する基礎技術の開発、特に、高速光軸制御や可変形状光学系の技術開発やトラッキング撮像に関する応用システムの開発を行っている。

% 高速三次元形状計測や高速質感計測など、並列処理に基づく高速画像処理技術 (理論、アルゴリズム、デバイス) 開発とその応用システムの実現を目指すシステムビジョンデザインの研究では、特に高速画像処理システムの開発、高速性を利用した新しい価値を創造する応用システムの開発を行っている。

% 実世界における新たな知覚補助技術並びにそれに基づく新しい対話の形の創出を目指すアクティブパーセプション技術の構築とその応用に関する研究では、特に各種高速化技術を用いた能動計測や能動認識を利用した革新的情報環境・ヒューマンインタフェイスの開発を行っている。

 本研究室では、 センサ情報処理における並列処理を基盤として、高速・リアルタイム性を有するセンサ情報処理を高度に実現し、高速知能システムとして実装する研究について、本年度は、 以下2つの分野での研究開発を行った。また、各分野での新規産業開拓にも力を注ぎ、研究成果の技術移転、共同研究、事業化等を様々な形で積極的に推進している。

ダイナミックビジョンシステムの研究では、高速ビジョンや動的光学系に基づき実世界上の運動対象情報を計測条件にとらわれずに取得する基礎技術の開発、 特に、 高速光軸制御や適応光学系の技術開発やトラッキング撮像に関する応用システムの開発、実政界への応用展開を行った。

アクティブパーセプションの研究では、 共同研究を基盤とした検査システムの高度化を目指し、車両の自己位置を高精度に算出可能な技術を新規に開発しただけでなく、基盤技術を生かしたアクティブトラッキング技術による知覚補助技術を新規に開発し、新たな学際分野の確立・創出を行った。

これらの研究では、原則的に並列かつリアルタイムの演算構造を有する現実の物理世界と同等の構造を工学的に実現することを目指しており、そのことにより、 現実世界の理解を促すばかりでなく、従来のシステムをはるかに凌駕する性能を有する高速知能システムを生み出すことができ、結果として、まったく新しい情報システムを構築することが可能となる。


\section{データ科学研究部門 教員研究活動}

\subsection{計算機を介した人と生態系のインタラクションの研究(小林 博樹)}

本研究室は計算機を介した人と生態系のインタラクションの研究の行っている。これまで人間を対象とした知能情報学の見地を、多様で複雑な実世界の生物・環境・地理学・獣医学領域へ応用・発展させる研究である。研究内容はコンピュータ科学、環境学、メディアアート、など多岐に亘っており、特に、計算機を介した人と生態系のインタラクションHCBI(Human-Computer-Biosphere Interaction)の概念を情報学分野で発表し、このテーマを中心に、環境問題の解決を目的として、国内外で研究活動を独自に行ってきた。古典的なコンピュータ科学では、HCI(Human-Computer Interaction)が主要な研究領域の1つとなるが、本研究室はこの研究領域を地球環境にまで拡大すべく、人間と生態系の調和あるインタラクションを実現するシステムを提案し「時空間スケールの大きい環境問題を自律的に解決する情報基盤技術」として、そのフィールドでの実証実験を試みている。つまり、コンピュータ科学の分野では人間が活動する地理空間を対象とした研究が中心であったが、本研究室は人間が活動していない、情報通信技術の応用が困難な地理空間を対象にした情報デザインと野生動物IoTの研究を行っている。このように本研究室は、情報工学をベースとして、特に計算機を用いて生態系と人間のインタラクションを専門として実績をあげている。2022年度から科学技術振興機構の創発的研究支援事業として業務を実施している。
 
%  \begin{雑誌論文}{1}
 \bibitem{kobayashi1-1}
 Wenjing Li, Xiaodan Shi, Dou Huang, Xudong Shen, Jinyu Chen, Hill Hiroki Kobayashi, Haoran Zhang, Xuan Song, Ryosuke Shibasaki,  "PredLife: Predicting Fine-Grained Future Activity Patterns ", IEEE Transactions on Big Data 9(6) 1658-1669.

 \bibitem{kobayashi1-2}
 Wenjing Li, Haoran Zhang, Jinyu Chen, Peiran Li, Yuhao Yao, Xiaodan Shi, Mariko Shibasaki, Hill Hiroki Kobayashi, Xuan Song 0001, Ryosuke Shibasaki,  "Metagraph-Based Life Pattern Clustering With Big Human Mobility Data", IEEE Trans. Big Data 9(1) 227-240.
 
 \end{雑誌論文}

 \begin{査読付}{1}
 \bibitem{kobayashi2-1}
 Usman Haider, Muhammad Hanif, Hill Hiroki Kobayashi, Laxmi Kumar Parajuli, Daisuké Shimotoku, Ahmar Rashid, Sonia Safeer, "Bioacoustics Signal Classification Using Hybrid Feature Space with Machine Learning", Proceedings of 15th International Conference on Computer and Automation Engineering(ICCAE), 2023.  

 \bibitem{kobayashi2-2}
 Leo Uesaka, Ambika Prasad Khatiwada, Daisuk\'e Shimotoku, Laxmi Kumar Parajuli, Manish Raj Pandey, Hill Hiroki Kobayashi, "Applications of Bioacoustics Human Interface System for Wildlife Conservation in Nepal", Proceedings of 2023 International Conference on Human-Computer Interaction (HCII 2023), 2023.  

 \end{査読付}



% %半ページから1ページが文量

\subsection{大規模グラフニューラルネットワークの理論と応用(鈴村 豊太郎)}

% 本節では2022年度の鈴村豊太郎の研究活動について報告する。 鈴村は、 グラフ構造に対するニューラルネットワークを用いた表現学習 Graph Neural Network (以下、GNNと呼ぶ)の基礎研究及び応用研究に取り組んでいる。 グラフ構造は、 ノードと、 ノード同士を接続するエッジから構成されるデータ構造である。 インターネット上における社会ネットワーク、 購買行動、 サプライチェーン、 金融における決済データ、 交通ネットワーク、 蛋白質相互作用・神経活動・DNAシーケンス配列内の依存性、 物質の分子構造、 人間の骨格ネットワーク、 概念の関係性を表現した知識グラフなど、 グラフ構造として表現できる応用先は枚挙に暇がない。
% \par
% 当該研究領域における研究として、時系列・動的に変化する大規模グラフに対するGNNモデルの研究を行った。
% 動的グラフに対応するGNNモデルはすでに数多く提案されているが、いずれも短期的なデータの変化しか考慮されておらず、実世界で扱われている長期的なグラフデータでは長期的なコンテキストを捉えることができない問題が潜在的に存在していた。この問題に対して、時間幅が非常に長いグラフデータの性質も捉えることができる Spectral Waveletを提案した(AAAI 2023 \cite{aaai-deft}, Transactions on Machine Learning Research (TLMR) \cite{feta})。また、知識グラフ上で足りない関係性を補完する手法を評価する方法として トポロジカルデータ解析(Topological Data Analysis) における Persistent Homologyの概念を用いて効率的に評価する手法を提唱した(WWW 2023 \cite{kg-kp})。
% % 当該研究領域における研究として、時系列・動的に変化する大規模グラフに対するGNNモデルの研究を行った。実世界では時間幅が非常に長いデータを扱うこともあるが既存の動的グラフへのGNNの研究ではそのような点を考慮していない。この問題に対して、学習可能な Spectral Waveletを提案し、AAAI 2023 \cite{aaai-deft}, WWW 2023 \cite{deft}、 TLMR \cite{feta}に採択された。

% GNNに関する応用研究も進めている。金融領域においては不正検出に関するGNNモデルの検証を行い、取引ネットワークをヘテロジニアスなグラフ構造に拡張することによりモデル性能の向上を達成した(KDD'22 MLG Workshop \cite{eth-gnn})。マテリアルズ・インフォマティクスの分野においては、情報基盤センターの芝隼人先生とガラス物質の形成過程モデルに対して高精度なGNNモデルを提案し\cite{botan}、また当該分野における本質的な問題に対する手法として、インバランスなデータの問題を解消するための手法 \cite{xsig-limin}および外挿のためのモデル構築を行った\cite{xsig-takashige}。E-Commerceの領域においては知識グラフを用いた商品推薦手法を提案した (ACM SIGIR 2023 \cite{sigir})。

% また、理論モデルの実世界への検証と応用サイドから意味のある研究テーマを発掘するため、企業との共同研究とも進めている。まず、自動車の走行軌跡データから次の位置や経路を予測し、ロケーションリコメンデーションなどに応用するための手法をトヨタ自動車と探求している。走行軌跡データは緯度・経度及び時刻のシーケンスデータとなるが、それを用いると運転行動パターンを捉えることができる。
% % まずシーケンスデータからグラフ構造を構築し、そのグラフ構造からGraphormerというニューラルネットワークモデルを走行軌跡データのパターンを捉えられるようなニューラルネットワークのモデルを提案した。
% この行動パターンを捉えるために、シーケンスデータをグラフ構造として表現し、Graphormerをベースにした新たなモデルを提案し、他の既存手法よりも高い精度でパターンを予測することを確認した (ECML-PKDD \cite{stgtrans} 査読中)。
% % この新たなモデルを他の既存手法と比較し、より高い精度で走行パターンを予測する事を確認した。本研究の成果を ECML-PKDD \cite{stgtrans}に提出した。
% 来年度はモビリティにおける様々な領域に応用できるように、走行軌跡データや実世界の地図データなどから事前学習モデルを構築する予定である。また、その他に都市全体の二酸化炭素排出量を抑制するために交通流を分散するための手法をこれらの事前学習モデルと深層強化学習を用いて設計・実装する予定である。

% また、エス・エム・エス社との共同研究では、介護や医療領域における人材紹介の推薦システムに関する研究を行った。超高齢化社会に突入する中、介護や医療領域における人材不足は深刻であり、より精度の高い人材マッチングが不可欠である。この問題に対して、深層強化学習を用いた人材マッチング数の最適化手法を提案し、特定の求職者・事業者側に偏ってしまう従来の推薦・マッチング手法に対して、偏りを解消できることを確認した (人工知能学会\cite{sms} 6月発表予定)。来年度に関しては更に実データでの検証を進め、企業側での要望を取り入れ、実ビジネスが持つ制約条件を取り入れた最適化モデルを提案していく予定である。また、モデルにおいて求職者と求人側での動的な二部グラフの関係性及び知識グラフを用いてより精度高いモデルを構築していく予定である。

%  日本経済新聞社(以下、日経)との共同研究も2022年10月から開始している。日経ではニュースサイトにおける記事に対してより高度な機械学習による推薦システムを目指しており、2022年度は日経側での問題設定やデータの理解を図り、研究テーマの設定に主に取り組んだ。2023年度は推薦問題や広告配信など様々な領域に応用できるように、ユーザーの行動モデルを統一的に表現する事前学習モデルを構築するべく、自己教師付き学習 (Self-Supervised Learning)を用いた手法を設計・実装する予定である。

%  また、GNNの概要と最新研究動向に関する記事を人工知能学会誌\cite{jsai-gnn}に寄稿し、Federated Learning(連合学習)の英語書籍向けに Federated Learningを用いた金融不正検知に関する手法を執筆した\cite{fl-book}。mdxプロジェクトに関する第一弾の国際学会論文として IEEE CBDCom\cite{mdx}にて論文発表を行った。


% % How Expressive are Transformers in Spectral Domain for Graphs? \cite{feta}
% % Learnable Spectral Wavelets on Dynamic Graphs to Capture Global Interactions \cite{deft}
% % Can Persistent Homology provide an efficient alternative for Evaluation of Knowledge Graph Completion Methods? \cite{kg-kp}


% % Spatio-Temporal Meta-Graph Learning for Traffic Forecasting \cite{megacrn}

% % Ethereum Fraud Detection with Heterogeneous Graph Neural Networks \cite{eth-gnn}

% % Federated Learning for Collaborative Financial Crimes Detection \cite{fl-book}



% %  本節では2021 年度の鈴村豊太郎の研究活動について報告する。 2021年4月に本学に着任し、 グラフ構造に関するニューラルネットワークを用いた表現学習 Graph Neural Network (以下、GNNと呼ぶ)の基礎研究及びその様々な応用研究に取り組んでいる。 グラフ構造は、 ノードと、 ノード同士を接続するエッジから構成されるデータ構造である。 インターネット上における社会ネットワーク、 購買行動、 サプライチェーン、 金融における決済データ、 交通ネットワーク、 蛋白質相互作用・神経活動・DNAシーケンス配列内の依存性、 物質の分子構造、 人間の骨格ネットワーク、 概念の関係性を表現した知識グラフなど、 グラフ構造として表現できる応用先は枚挙に暇がない。
% % \par
% % 当該研究領域において、時系列・動的に変化する大規模グラフに対するGNNモデルの研究を行った。分散計算環境においてスケールするGNNモデルを提唱し、その成果は高性能計算分野におけるトップカンファレンスSC2021\cite{suzumura-sc2021}に採択された。 また、金融領域における不正検知手法として、TransformerアーキテクチャをベースにしたGNN手法を提案し、国際会議 IEEE SMDS 2021\cite{suzumura-smds21}に採択された。 また、 GNNに関する招待講演\cite{suzumura-canon2021}を行った。

% % これらの研究に続いて、推薦システムへのGNNモデルに関する研究を開始している。実データ・実問題に基づいた、社会実装を見据えた研究を進めるべく、医療・介護領域における人材推薦としてエス・エム・エス社、自動車における経路推薦としてトヨタ社と共同研究を2023年4月から本格的に開始する。また、国立研究開発法人物質・材料研究機構NIMSが主導する「マテリアル先端リサーチインフラ」プロジェクトの本学拠点の一貫で、材料情報科学 Materials Informaticsへの研究も開始している。
% %  データ科学・データ利活用のためのクラウド基盤 mdx プロジェクトにおいて、 今年度は 課金付き運用開始に向けたシステム拡張、スポットVM、データ共有機構(Platform-as-a-Service)に向けた設計を進めた。 また、 mdxに関する講演活動を国内外において行った\cite{suzumura-axies2021、suzumura-nanotec2021、 suzumura-nci2021}。 mdxの論文においては、国際会議IEEE IC2E2022(10th IEEE International Conference on Cloud Engineering) に2022年3月末に投稿した。 
% % %\cite{suzumura-mdx2022}においても論文を公開した。


% 




% % 
% % bibitem を作る
% \begin{雑誌論文}{1}
% \bibitem{feta}
% Bastos, Anson, Abhishek Nadgeri, Kuldeep Singh, Hiroki Kanezashi, Toyotaro Suzumura, and Isaiah Onando Mulang, "How Expressive Are Transformers in Spectral Domain for Graphs?", Transactions on Machine Learning Research (TMLR) ISSN 2835-8856, Journal of Machine Learning Research, 2022.
% \end{雑誌論文}


\begin{査読付}{5}

\bibitem{mobgt}
Xiaohang Xu, Toyotaro Suzumura, Jiawei Yong, Masatoshi Hanai, Chuang Yang, Hiroki Kanezashi, Renhe Jiang, and Shintaro Fukushima. 2023. Revisiting Mobility Modeling with Graph: A Graph Transformer Model for Next Point-of-Interest Recommendation. In Proceedings of the 31st ACM International Conference on Advances in Geographic Information Systems (SIGSPATIAL '23). Association for Computing Machinery, New York, NY, USA, Article 94, 1–10.

\bibitem{glory}
Boming Yang, Dairui Liu, Toyotaro Suzumura, Ruihai Dong, and Irene Li. 2023. Going Beyond Local: Global Graph-Enhanced Personalized News Recommendations. In Proceedings of the 17th ACM Conference on Recommender Systems (RecSys '23). Association for Computing Machinery, New York, NY, USA, 24–34.

\bibitem{sigir}
Md Mostafizur Rahman, Daisuke Kikuta, Satyen Abrol, Yu Hirate, Toyotaro Suzumura, Pablo Loyola, Takuma Ebisu and Manoj Kondapaka, "Exploring 360-Degree View of Customers for Lookalike Modeling",  SIGIR'23 (The 46th International ACM
SIGIR Conference on Research and Development in Information
Retrieval) 

\bibitem{kg-kp}
Anson Bastos, Kuldeep Singh, Abhishek Nadgeri, Johannes Hoffart, Toyotaro Suzumura, Manish Singh,
"Can Persistent Homology provide an efficient alternative for Evaluation of Knowledge Graph Completion Methods?"
In proceedings of The Web Conference (WWW), 2023.

\bibitem{job-aaai}
Waki, Satoshi, Toyotaro Suzumura, and Hiroki Kanezashi. "Optimizing Matching Markets with Graph Neural Networks and Reinforcement Learning." In Workshop on Recommendation Ecosystems: Modeling, Optimization and Incentive Design, AAAI 2024.

\bibitem{num-aaai}
Putra, Refaldi, and Toyotaro Suzumura. "On the Role of Numerical Encoding in Foundation Model of Sequential Recommendation with Sequential Indexing." In Workshop on Recommendation Ecosystems: Modeling, Optimization and Incentive Design. 2024, AAAI 2024


\bibitem{job-jsai}
脇聡志, 鈴村豊太郎, 金刺宏樹, 華井雅俊, 小林秀. "強化学習によるマッチング数を最大化するジョブ推薦システム." 人工知能学会全国大会論文集 第 37 回 (2023), 一般社団法人 人工知能学会, 2023.
\end{査読付}

% \begin{著書}{2}

% \bibitem{fl-book}
% Toyotaro Suzumura, Yi Zhou, Ryo Kawahara, Nathalie Baracaldo, Heiko Ludwig,
% "Federated Learning for Collaborative Financial Crimes Detection",
% Ludwig, H., Baracaldo, N. (eds) Federated Learning. Springer, Cham. https://doi.org/10.1007/978-3-030-96896-0\_20

% \bibitem{jsai-gnn}
% 鈴村 豊太郎, 金刺 宏樹, 華井 雅俊, グラフニューラルネットワークの広がる活用分野, 人工知能, 2023, 38 巻, 2 号, p. 139-148, 公開日 2023/03/02, Online ISSN 2435-8614, Print ISSN 2188-2266, https://doi.org/10.11517/jjsai.38.2\_139

% \end{著書}

% \begin{招待講演}{3}
% \bibitem{fugaku} 鈴村豊太郎「夢の形~未来のコンピュータ~」(パネリスト)、スーパーコンピュータ「富岳」第2回成果創出加速プログラムシンポジウム「富岳百景」、2022年12月21日
% \bibitem{jsse1}鈴村豊太郎,「データ活用社会創成プラットフォームmdxおよび 大規模グラフニューラルネットワーク」, 第34回CCSEワークショップ「原子力材料研究開発におけるDX推進の現状と将来:原子力材料研究開発の革新と新展開」, 2023年2月24日
% \bibitem{jsse2}鈴村豊太郎,「人工知能の最先端研究に迫る ~大規模グラフニューラルネットワークの世界へ」, 東京大学柏キャンパス一般公開2022、特別講演会、2022年10月22日
% \bibitem{rakuten}Toyotaro Suzumura, “How Will Data and AI Change the World?”,  Rakuten Optimism Conference, Tokyo, Japan, September 29, 2022
% \bibitem{rccs}Toyotaro Suzumura,  “Large-Scale Graph Neural Networks for Real-World Industrial Applications”, The 5th R-CCS International Symposium, Kobe Japan, February 7, 2023
% \bibitem{france}Toyotaro Suzumura,  “Large-Scale Graph Neural Networks for Real-World Industrial Applications”, International Workshop “HPC challenges for new extreme scale application” held by French Alternative Energies and Atomic Energy Commission, Paris, France, March 6, 2023
% \bibitem{barcelona}Toyotaro Suzumura,  “Large-Scale Graph Neural Networks for Real-World Industrial Applications”, Barcelona Supercomputing Center, Barcelona, Spain, March 10, 2023
% \end{招待講演}



% \begin{発表}{1}
% \bibitem{sms}
% 脇 聡志, 鈴村 豊太郎, 金刺 宏樹, 小林 秀, "強化学習によるマッチング数を最大化するジョブ推薦システム." 第37回人工知能学会全国大会 (2023),  一般社団法人 人工知能学会, (2023年6月発表予定)

% \end{発表}


\subsection{データセンタハードウェアへのソフトウェア脆弱試験の適応(空閑 洋平)}

% 現在のデータセンタ環境では、機械学習やニューラルネットワークの学習,推論を高速化する専用アクセラレータが広く使用されるようになった。専用アクセラレータを用いた計算環境は、既存のCPUを中心に構成されていたソフトウェア環境に比べて、プロセッサやデバイスドライバ、デバイス間通信が専用に設計され、CPUをバイパスしてデバイス間で直接データ通信されるため、デバイスのデータ通信の把握や可視化が困難なブラックボックス化が進んでいる。今後、専用アクセラレータを中心とした次世代のデータセンタ環境では、CPUをバイパスするデバイス間通信が増加することで、セキュリティ監視や脆弱性試験、管理手法、データ通信内容の可視化手法といった、普段CPU環境で実施している運用課題が顕在化すると考えられる。

% 今年度は、CPUからデータセンタハードウェアを直接操作するために設計した独自Remote DMA機能を用いた研究を実施した\cite{ykuga39987672,ykuga41835070,ykuga37056192}。
% 特に、今年度はソフトウェアでプログラム可能なメモリデバイスを提案し、ソフトウェアでアクセラレータやストレージのデータ通信内容の観測と書き換えが可能なことを確認できたため、来年度はアクセラレータに対する脆弱性試験を予定している。
% また、昨年から引き続き、CPUをバイパスするNIC型ネットワークルータアーキテクチャの検討を実施し、今年度は経路表のみをCPUで処理するハイブリットアプローチを提案している\cite{ykuga36919054}。
% mdxの機能高度化に関する研究については、データ転送機能の高性能化手法の検討と、mdx上でのkubernetes基盤構築に関する機能開発について招待論文で報告した\cite{ykuga41835081,ykuga41534619}。
% その他の活動としては、東京大学のZoomデータを用いて、広域ネットワーク品質解析の手法を提案した\cite{ykuga40356877}。

% \begin{査読付}{1}
\bibitem{ykuga43404131}
Ryo Nakamura, Yohei Kuga, Multi-threaded scp: Easy and Fast File Transfer over SSH, Practice and Experience in Advanced Research Computing, 2023年7月.

\end{査読付}

\begin{雑誌論文}{1}
\bibitem{ykuga45871761}
空閑洋平, 中村遼, 遠隔会議システムの計測データを用いたネットワーク品質計測, 情報処理学会論文誌, 65, 3, pp646-655, 2024年3月.

\end{雑誌論文}

\begin{招待講演}{1}
\bibitem{ykuga45871732}
空閑洋平, データセンタハードウェアへのソフトウェア脆弱試験の適応, Society 5.0時代の安心・安全・信頼を支える基盤ソフトウェア技術の構築, 2024年3月.

\end{招待講演}

\begin{招待論文}{1}
\bibitem{ykuga45871752}
河原大輔, 空閑洋平, 黒橋禎夫, 鈴木潤, 宮尾祐介, LLM-jp: 日本語に強い大規模言語モデルの研究開発を行う組織横断プロジェクト, 自然言語処理, 31, 1, pp266-279, 2024年.

\end{招待論文}

\begin{受賞}{1}
\bibitem{ykuga43010880}
空閑洋平, 山下記念研究賞, 情報処理学会(OS研究会), 2024年3月.

\bibitem{ykuga43010877}
空閑洋平, 山下記念研究賞, 情報処理学会(IOT研究会), 2024年3月.

\bibitem{ykuga43010874}
空閑洋平, 中村遼, 藤村記念ベストプラクティス賞, 情報処理学会(IOT研究会), 2023年7月.

\bibitem{ykuga9999}
中村遼, 空閑洋平, 明石邦夫, Most Innovative for HPC Uses Award, Data Mover Challenge 2023, 2024年2月.

\end{受賞}


\subsection{Research on the use of ICT for biodiversity conservation(Parajuli Laxmi Kumar)}

% I have been working with Professor Kobayashi on the use of space technology for biodiversity conservation in Nepal. During fiscal year 2022, I discussed with the research officers in National Trust for Nature Conservation (NTNC), Nepal about the possibility of incorporating high technology for nature conservation in Nepal. We also discussed with the locals in Nepal about their needs and how technological intervention is necessary to address their problems. A technical cooperation request was submitted by the NTNC on the use of space technology such as GPS radio collars, drones, infrared sensors, bioacoustics, camera traps and LiDAR technology. I was also involved in writing a perspective paper (Uesaka et al., 2023) in collaboration with the researchers in the laboratory of Professor Kobayashi and two scientists in NTNC, Nepal. 
 

% \input{Kumar/ITCannual-list-Kumar}

\subsection{第一原理計算とデータ科学・機械学習による物質科学研究(河村 光晶)}

物質を構成する電子や原子に対して、基本法則となる(相対論的)量子力学や統計物理学に基づく理論的研究と、様々な環境(温度、圧力等)やプローブ(電子線、可視光、X線、中性子線、物理量測定等)における実験的研究の両者を協調的に行う事で、既存の現象の理解や新たなる物質探索をより効果的に進めることができる。
実験との比較を行うにあたり、物性物理学の理論を多様な組成$\cdot$構造を持つ現実の物質に適用するためには計算機によるシミュレーションが不可欠となり、スパコンやmdxのような高速$\cdot$大規模な並列計算資源およびデータ格納環境が利用される。
我々はそのような大規模計算機における物質科学シミュレーションの手法やプログラムの開発、およびそれを実際の物質$\cdot$現象に適用する研究を行っている。

二硫化モリブデン($\textrm{MoS}_2$)は化学的$\cdot$機械的安定性や高い電子易動度により、電子デバイスや触媒として用いられている。特に脱硫触媒としての性能において重要となるのが大気中での表面の硫黄欠陥の安定性やその周囲の構造変化である。
大気圧中での物質表面での特定の元素の周りの状況(化学的環境)を測定することは容易ではないが、間接的にそれを知ることができる方法の一つがX線光電子分光(XPS)である。この方法では、内核電子の束縛エネルギーが元素によって大きく異なることを利用して、元素選択的な信号を得ることができる。XPSを$\textrm{MoS}_2$に適用した結果では、硫黄(S)、モリブデン(Mo)ともに表面での束縛エネルギーの低下が観測された~\cite{kawamura_mos2}。これは両元素とも周囲の電子の量が増えていることを示唆している。またいくつかの表面構造モデルでの第一原理シミュレーションによれば、S原子が欠損したモデルにおいて計算で得られた束縛エネルギーが実験地と定量的に一致しており、欠損したアニオンSから両元素に電子が供給されるという機構を提案した。

極低温で金属の電気抵抗が消失する超伝導は、量子力学的性質が巨視的なスケールで発現する特異な現象であり、新たな超伝導物質の探索や発現機構の解明のための研究が行われている。
そのようなものの一つとして、既知の超伝導体$\textrm{La}_2\textrm{IRu}_2$のルテニウム(Ru)をオスミウム(Os)で置き換えた新規物質$\textrm{La}_2\textrm{IOs}_2$が合成され、この置換により超伝導転移温度($T_c$)が4.8 Kから12 Kへと大幅に変化することが観測された~\cite{kawamura_la2ios2}。RuとOsは同じ属の元素であり、非常に似た性質を持つため単体での$T_c$はほぼ同じであるが、この化合物ではLaとの軌道混成のために電子状態が大きく異なることを第一原理計算により明らかにした。またこれらの物質では磁場に対して超伝導が頑強であることも興味深い。
第一原理計算からの$T_c$予測による新規物質探索にも注目が集まっており、そのための手法開発や、現時点での理論の精度の検証もおこなっている~\cite{kawamura_yitp}。
超伝導に限らずあらたな熱電材料・構造材料の理論的探索として、$MAX$相の網羅的構造安定性解析を行った~\cite{kawamura_max}。
$MAX$相とは、d電子の少ないスカンジウム、チタン等の遷移金属元素($M$)、d電子の多い金属元素やアルミニウム等の典型元素($A$)と炭素または窒素($X$)からなる層状化合物の総称であり、構成元素の組み合わせや組成比($M_{n+1}AX_n$)により多様な物質が提案できる。
我々は密度半関数理論に基づく第一原理構造最適化と、格子振動解析を行い未だ合成されていないいくつかの$MAX$相とその長周期構造の提案を行い、フェルミ面の形状からその長周期構造の発現機構を議論した。

このように密度汎関数理論に基づく第一原理計算による記述が効果的な系がある一方で、そのような手法では捉えきれない研究対象が存在する。
強相関電子系とよばれる物質群では、非従来型超伝導や量子スピン液体といった全く新しい物理現象が発現することがあるが、それらの機構を解明するためのシミュレーションには厳密対角化や量子モンテカルロ法といったより高度な計算手法が必要となる。
第一原理計算の結果を前処理として、単純なモデル化を経て高度な解析を行うためのぽうろグラムパッケージの開発を行っており、
厳密対角化を行う$\mathcal{H}\Phi$~\cite{kawamura_hphi,kawamura_hpci3}と多変数変分量子モンテカルロ法をおこなうmVMC~\cite{kawamura_hpci2}がある。
これらのプログラムは東京大学物性研究所のソフトウェア高度化プロジェクトのなかで開発が行われた~\cite{kawamura_hpci1, kawamura_kotai}。

% % \begin{雑誌論文}{3}

% \bibitem{osada.pccp.24.21705}
% Wataru Osada, Shunsuke Tanaka, Kozo Mukai, Mitsuaki Kawamura, YoungHyun Choi, Fumihiko Ozaki, Taisuke Ozaki and Jun Yoshinobu, 
% "Elucidation of the atomic-scale processes of dissociative adsorption and spillover of hydrogen on the single atom alloy catalyst Pd/Cu(111)",
% Physical Chemistry and Chemical Physics 24, 36, 21705-21713, 2022.

% \bibitem{hirai.jacs.144.17857}
% Daigorou Hirai, Keita Kojima, Naoyuki Katayama, Mitsuaki Kawamura, Daisuke Nishio-Hamane, and Zenji Hiroi,
% "Linear Trimer Molecule Formation by Three-Center-Four-Electron Bonding in a Crystalline Solid RuP",
% Journal of the American Chemical Society 144, 39, 17857-17864, 2022.

% \bibitem{koshoji.prm.6.114802}
% Ryotaro Koshoji, Masahiro Fukuda, Mitsuaki Kawamura, Taisuke Ozaki,
% "Prediction of quaternary hydrides based on densest ternary sphere packings",
% Physical Review Materials 6, 11, 114802.1-9, 2022.

% \end{雑誌論文}



\subsection{野生動物ワイヤレスセンサネットワークと時空間行動分析に関する研究(川瀬 純也)}
%  本研究室では、野生動物装着型ワイヤレスセンサーネットワーク(以下:野生動物 WSN)機構による自然環境等でのデータ収集手法の開発と、それによって得られるデータの解析手法についての研究を行っている。人間が容易には侵入できないエリアでの継続的なデータ収集機構として種々の社会問題の解決に寄与することを目指している。
 
% 2022年度は、コロナ禍において中断を余儀なくされていた家畜等の動物による各種実験・データの収集を再開するべく、その環境の再構築に重点を置いて研究活動を進めてきた。動物による実験は専門家の指導の下、実験内容や対象となる動物に合わせた各種設備・環境を整えて行う必要があるため、その環境の構築と整備は慎重かつ丁寧な確認を繰り返して行う必要がある。2022年夏より、北海道のある乳牛農家に実験の協力を依頼すべく、地元の家畜用機器を扱う業者と協力しながら検討・議論を進めている。

%  また、野生動物を対象に含む時空間行動の分析の手法として、配列アライメント手法を用いた類型化手法に注目し、研究を進めている。配列アライメント手法は、もともとはバイオインフォマティクスで用いられる手法であり、編集距離という概念と動的計画法によって、文字列間の類似度を定量的に算出し、類型化する手法である。筆者は博士論文時からこの手法の時空間行動分析への応用手法の検討を行ってきた。野生動物の時空間行動は、その意図や目的などを明示的に把握することができず、GNSSデータなどの移動データから推測するしかない。多種多様で大量な移動データを分析する上で、定量的な類型化手法は重要となる。現在活用可能な移動データを用いて手法の検討を進めており、2023年中に順次報告を行う予定である。
% \input{Kawase/ITCannual-list-Kawase}

\subsection{材料データの収集および解析のための情報システム基盤に関する研究(華井 雅俊)}

本節では、2023年度の華井雅俊の研究活動について報告する。
今年度は主に大規模材料科学データの収集および解析のための情報システム基盤に関する研究開発を主に取り組んだ。

近年、機械学習分野の社会的な盛り上がりやマテリアルズインフォマティクスの発展などから、材料科学データの重要性がますます強調されている。
材料の開発研究においてデータはおおよそ2つに分類され、1つは理論計算によって生み出されるシミュレーションデータ、もう1つは実際の材料実験装置 (電子顕微鏡や放射光装置) から得られる実験データである。
今日における個々の材料研究開発において、それらの相互的なデータ解析・データ同化は不可欠であり広く一般的に行われている。一方で、実験とシミュレーションを統合した大規模なデータ収集や利活用、更に汎用的に利用可能な大規模データセット構築などは未だ決定的な提案がなく、課題が多い。
機械学習やグラフニューラルネットワークネットワークなどよりハイレベルなデータ駆動型研究を支える基礎インフラとして、今年度の本研究では材料用大規模データ集積・解析基盤を構築し主にシステム面の問題解決に注力した。

具体的には今年度にかけて東京大学工学系研究科と共同で、東京大学の共用材料実験施設の運用プロジェクト (ARIM) 向けにARIM-mdxデータシステムを開発し~\cite{hanai-arim-mdx}、一般ユーザーの利用を開始した。
8月末の運用開始から3月末までに学内・学外および企業ユーザーを含む237ユーザーに達し、着実なスタートを切ることができた。
東大の武田クリーンルームユーザーへの展開、北大・九大・産総研を含む学外展開を進めており来年度以降はさらなるユーザー獲得に注力する。

本データシステムは主に3つの機能からなり、1つは総量3PBの超大容量クラウドストレージ、2つめはIoTデバイスによる直接データ転送、3つめはK8sによる分散コンテナ計算環境をベースとした Jupyter, VSCode, RemoteDesktop環境である。
特に2つ目のIoTデバイスにおいては、これまでデータ取りだしのクラウド化が難しかった非ネットワーク計測装置に、独自開発のIoTを接続し経由することで、クラウドストレージ環境に直接データ転送を行うことができるようになった。
本IoT上で動作するデータ転送アルゴリズムはユーザー認証および耐故障性をもち、各ユーザーがそれぞれの貴重な実験データを安全にクラウドストレージに通信することが可能となった。
技術的な詳細は\cite{hanai-UCC}にまとめ、IoT/クラウドに関する国際会議にて発表を行った。
本IoTデバイスの技術は、東京大学TLOにその社会的なニーズを認められ、特許出願 ~\cite{hanai-iot} に結びついたため、来年度、パートナー企業との機能の充実および社会実装を積極的に進める。




% \cite{hanai-xiaohang}
% \cite{hanai-iot}
% \cite{hanai-utac}
% \cite{hanai-arim-mdx}




% 本節では、2022年度の華井雅俊の研究活動について報告する。グラフニューラルネットワーク(Graph Neural Network, GNN) とその物性予測問題への応用に関する研究に取り組んでいる。
% 電池、半導体、触媒、医薬品などの材料開発・材料研究の全般において、膨大にある候補材料のさまざまな物性を比較解析することが不可欠であるが、それら候補全てを実際に作り検証することは現実的でない。そのため分子構造などの比較的簡単に得られる物質情報から目的の物性を予測・計算することが重要である。近年では、分子構造(グラフ)データとグラフニューラルネットワークを利用した物性値予測モデルの研究が盛んになってきている。2021年度に引き続き、Stanford Universityが取りまとめるOpen Graph Benchmark (OGB) やCMUとFacebookが主導するOpen Catalyst Project (OCP) などの物性予測問題ベンチマークが機械学習系研究コミュニティで取り上げられ、ますますの盛り上がりを見せている。

% 2022年度は、GNNを用いた物理問題へのアプローチに関して大きく2の方向性から取り組んでいる。1つは、既存GNNモデルを物理の問題へ応用した際に現れる機械学習手法の限界に関する研究である。機械学習で典型的な、画像処理や自然言語処理では注力されないが物理の問題では非常に重要となる外挿予測とデータの不均衡性に関して特に取り組んだ~\cite{xsig-limin,xsig-takashige}。
% もう1つは対象の物質により注力した応用研究である。具体的には、ガラスのダイナミクスの予測問題に着手した~\cite{botan}。 ガラスの振る舞いをグラフを用いてモデル化しGNNを用いることで、分子動力学などのシミュレーション結果を詳細に予測した。
% また、その他のGNN応用とも共通の課題として鈴村研究室メンバーとの共同研究も行っており、例えば交通システムの問題に関して研究を行った~\cite{stgtrans}。

% また、業務では情報基盤センターが進めるmdxに関して、物性研究や材料開発で得られるデータの利活用を進めている。本年度は物性データに特化したペタスケールストレージをmdxに連携させるシステムを設計し導入を行った。

% % 一般に、ある物性値が広範囲な材料群に対し既知である場合予測モデルを構築することが可能となるが、しかし一方で、多くの物性値においては既知である材料が少数であり学習データが不足しているため、実用精度の予測モデルを構築することは難しい。同一の物性であってもパラメータや実験条件が共通化されていないと予測モデルの構築は難しいことが知られ、既存の物性予測の研究では、共通の条件で整理された大規模データが主に利用される(例えば、上のコンペティションなど)。小規模に限定されるデータ、例えば計算コストの膨大なシミュレーション値や実験データ、において、機械学習の利用は限定的であり、大きな研究課題の1つとなっている。

% % 我々の研究チームはこのような少規模データに着目し研究を開始した。2021年度下半期は新手法提案への準備としてデータの収集に注力し研究を行った。機械学習分野や材料研究分野で用いられるオープンデータに加え、同学の工学部の研究チームへコンタクトし、スパコンスケールの計算資源を利用し得られた高価なシミュレーション値や実際の実験データに関してヒアリングを行い、データ収集を開始した。
% % また、本部門で開発のすすめるmdxにおいては材料系研究への利用促進を行っており、本研究の中間報告として第20回ナノテクノロジー総合シンポジウムにて発表し、IEEE IC2E 2022への投稿論文にて材料系研究におけるクラウド基盤の利活用をまとめた。

% % % 2021年度は主に、分野の調査と


% % 本節では、2021年度の華井雅俊の研究活動について報告する。2021年9月の本学着任から、グラフニューラルネットワークとその物性予測問題への応用に関する研究に取り組んでいる。

% % 電池、半導体、触媒、医薬品などの材料開発・材料研究の全般において、膨大にある候補材料のさまざまな物性を比較解析することが不可欠であるが、それら候補全てを実際に作り検証することは現実的でない。そのため分子構造などの比較的簡単に得られる物質情報から目的の物性を予測・計算することが重要である。近年では、分子構造(グラフ)データとグラフニューラルネットワークを利用した物性値予測モデルの研究が盛んになってきている。特に2021年度はStanford Universityが取りまとめるOpen Graph Benchmark (OGB) やCMUとFacebookが主導するOpen Catalyst Project (OCP) などの機械学習系研究コミュニティのコンペティションで物性予測問題が取り上げられた初めての年であった。

% % 一般に、ある物性値が広範囲な材料群に対し既知である場合予測モデルを構築することが可能となるが、しかし一方で、多くの物性値においては既知である材料が少数であり学習データが不足しているため、実用精度の予測モデルを構築することは難しい。同一の物性であってもパラメータや実験条件が共通化されていないと予測モデルの構築は難しいことが知られ、既存の物性予測の研究では、共通の条件で整理された大規模データが主に利用される(例えば、上のコンペティションなど)。小規模に限定されるデータ、例えば計算コストの膨大なシミュレーション値や実験データ、において、機械学習の利用は限定的であり、大きな研究課題の1つとなっている。

% % 我々の研究チームはこのような少規模データに着目し研究を開始した。2021年度下半期は新手法提案への準備としてデータの収集に注力し研究を行った。機械学習分野や材料研究分野で用いられるオープンデータに加え、同学の工学部の研究チームへコンタクトし、スパコンスケールの計算資源を利用し得られた高価なシミュレーション値や実際の実験データに関してヒアリングを行い、データ収集を開始した。
% % また、本部門で開発のすすめるmdxにおいては材料系研究への利用促進を行っており、本研究の中間報告として第20回ナノテクノロジー総合シンポジウムにて発表し、IEEE IC2E 2022への投稿論文にて材料系研究におけるクラウド基盤の利活用をまとめた。

% % % 2021年度は主に、分野の調査と




% % \begin{雑誌論文}{1}

% \bibitem{gnn-glass}
% Hayato Shiba, Masatoshi Hanai, Toyotaro Suzumura, and Takashi Shimokawabe, "BOTAN: BOnd TArgeting Network for prediction of slow glassy dynamics by machine learning relative motion." The Journal of Chemical Physics 158, no. 8, 084503, 2022.
% \end{雑誌論文}

% \begin{査読付}{3}

% \bibitem{xsig-limin-hanai}
% Limin Wang, Masatoshi Hanai, Toyotaro Suzumura, Shun Takashige, Kenjiro Taura, "On Data Imbalance in Molecular Property Prediction with Pre-training" xSIG 2023 (submitted)

% \bibitem{xsig-takashige-hanai}
% Shun Takashige, Masatoshi Hanai, Toyotaro Suzumura, Limin Wang, Kenjiro Taura, "Is Self-Supervised Pretraining Good for Extrapolation in Molecular Property Prediction?" xSIG 2023 (submitted)

% \bibitem{stgtrans-xiaohang}
% Xiaohang Xu, Toyotaro Suzumura, Jiawei Yong, Masatoshi Hanai, Chuang Yang, Hiroki Kanezashi, Renhe Jiang, Shintaro Fukushima, "Spatial-Temporal Graph Transformer for Next Point-of-Interest Recommendation", Machine Learning and Knowledge Discovery in Databases: European Conference, (ECML-PKDD), 2023 (submitted)

% \end{査読付}


%\subsection{生態音響システムを用いた生物調査(上坂 怜生)}

% 本節では、2022年度の上坂怜生の研究活動について報告する。本年度は生態音響システムによる高線量地域の鳥類の調査を行った。科学技術振興機構の創発的研究支援事業の一部として行っている本研究では、福島県浪江町の帰宅困難地域内での生態系の経年的変化を音響的手法で調査することを目的としている。当研究室では帰宅困難地域内に設置されたマイクにより過去複数年に渡って調査地の環境音を収集しており、生態学的な価値を持つデータが蓄積され続けている。本年度はこの生態音響データの解析に着手し始めた。実際に福島県浪江町の帰宅困難地域へ赴き、マイクの設置状況等を確認するといった定期的なメンテナンスも行った。本システムによって蓄積されている音響データには特にカラスや鶯といった鳥類の音声が多量に含まれているため、音声の特徴を抽出することでこれら鳥類の出現頻度を検証できるよう解析を進めている。

% また本年度は、生態音響による調査手法を国内のみならず発展途上国の生態系調査へと応用する可能性を鑑み、ネパールの自然保護機関であるNational Trust for Nature Conservation(NTNC)との議論を進めた。福島県の帰宅困難地域という社会的インフラ(電気やインターネット接続)のなかった場所に新たに作り上げた音響システムは、インフラ設備が整っていない発展途上国の広大な野外環境においても応用可能と考えられる。しかし、実現のためには現地の住民や政府機関を含めたステークホルダーの説得や現地研究者の情報技術リテラシー教育など、様々な問題をひとつずつ解決する必要があるため、入念な事前調査が必要である。本年度はネパールのチトワン国立公園を実際に訪れることで、どのような動物がどのような危機にさらされているか事前調査を行った。また、生態音響機材を実際にネパール国内の研究者とともに設置しながら議論を進めることで、生態音響的手法が保全活動に大きく貢献できることを確認した。ネパール国内での調査を福島県浪江町で行っている研究と並行して進めることで、生態音響システムによる生物調査のさらなる発展を促すことができると考えられる。
 
% % \begin{雑誌論文}{1}

% \bibitem{Uesaka2023HCII}
% Uesaka L, Khatiwada AP, Shimotoku D, Parajuli LK, Pandey MR, Kobayashi HH (2023). Applications of Bioacoustics Human Interface System for Wildlife Conservation in Nepal. Accepted for publication in Human Computer Interaction (HCI) International 2023.

% \end{雑誌論文}


\subsection{大規模言語モデルの研究とその応用(Li Zihui)}

Educational Large Language Models (LLMs): This research area focuses on enhancing the discovery of educational resources and the interpretability of transformer models in NLP education. We explored whether traditional methods could effectively identify high-quality study materials and proposed a transfer learning-based pipeline to improve resource discovery, especially for generating introductory paragraphs in educational content \cite{ireneli-3,ireneli-9}. Additionally, we assessed the potential of LLMs as tools for learning support, presenting a benchmark to evaluate their performance in various NLP tasks \cite{ireneli-5}. Our investigation also included the generation of concise survey articles in the NLP domain using LLMs. While GPT-created surveys were found to be more up-to-date and accessible than human-written ones, some limitations, such as occasional factual inaccuracies, were noted \cite{ireneli-11}. Furthermore, we explored the use of LLMs in teaching complex legal concepts through narrative methods \cite{ireneli-8}. Our LLM research works \cite{ireneli-1} have been noticed by \textbf{Nature News}, and I had the opportunity to be interviewed by a reporter, where I shared my insights on the impact of Large Language Models (LLMs) from the academia perspective \cite{ireneli-13}.

Medical Large Language Models (LLMs): In medical LLM research, we focus on enhancing LLMs' capabilities in medical question answering and teaching complex legal concepts via storytelling. Collaborating with MatsuoLab, we developed a framework combining knowledge graphs and ranking techniques to boost LLMs' effectiveness in medical queries \cite{ireneli-7}. We also introduced Ascle, a Python NLP toolkit for medical text generation \cite{ireneli-10}.

Other Benchmarks for NLP Open Questions: Our research extends to improving knowledge graph completion, evaluating LLMs in NLP problem-solving, and developing medical text generation tools. We investigated the enhancement of knowledge graph completion using node neighborhood data \cite{ireneli-4} and explored methods to better interpret transformer models by emphasizing crucial information \cite{ireneli-6}. Besides, we have been investigating other branches including graph methods for news encoding \cite{ireneli-2,ireneli-12}.



% % \begin{雑誌論文}{1}

% \bibitem{li2022neural}
% Irene Li, Jessica Pan, Jeremy Goldwasser, Neha Verma, Wai Pan Wong, Muhammed Yavuz Nuzumlalı, Benjamin Rosand, Yixin Li, Matthew Zhang, David Chang, R. Andrew Taylor, Harlan M. Krumholz, Dragomir Radev, "Neural Natural Language Processing for unstructured data in electronic health records: A review", Computer Science Review, volume 46, 2022.

% \end{雑誌論文}

% \begin{査読付}{1}

% \bibitem{feng2022diffuser}
% Aosong Feng and Irene Li and Yuang Jiang andRex  Ying, "Diffuser: Efficient Transformers with Multi-hop Attention Diffusion for Long Sequences", Proceedings of Thirty-Seventh AAAI Conference on Artificial Intelligence (AAAI), 2023.

% \end{査読付}

% \begin{発表}{1}

% \bibitem{li2023nnkgc}
% Zihui Li and Boming Yang and Toyotaro Suzumura, "NNKGC: Improving Knowledge Graph Completion with Node Neighborhoods", arXiv preprint, 2023.

% \end{発表}

\subsection{生態音響システムを用いた生物調査(上坂 怜生)}

% 本節では、2022年度の上坂怜生の研究活動について報告する。本年度は生態音響システムによる高線量地域の鳥類の調査を行った。科学技術振興機構の創発的研究支援事業の一部として行っている本研究では、福島県浪江町の帰宅困難地域内での生態系の経年的変化を音響的手法で調査することを目的としている。当研究室では帰宅困難地域内に設置されたマイクにより過去複数年に渡って調査地の環境音を収集しており、生態学的な価値を持つデータが蓄積され続けている。本年度はこの生態音響データの解析に着手し始めた。実際に福島県浪江町の帰宅困難地域へ赴き、マイクの設置状況等を確認するといった定期的なメンテナンスも行った。本システムによって蓄積されている音響データには特にカラスや鶯といった鳥類の音声が多量に含まれているため、音声の特徴を抽出することでこれら鳥類の出現頻度を検証できるよう解析を進めている。

% また本年度は、生態音響による調査手法を国内のみならず発展途上国の生態系調査へと応用する可能性を鑑み、ネパールの自然保護機関であるNational Trust for Nature Conservation(NTNC)との議論を進めた。福島県の帰宅困難地域という社会的インフラ(電気やインターネット接続)のなかった場所に新たに作り上げた音響システムは、インフラ設備が整っていない発展途上国の広大な野外環境においても応用可能と考えられる。しかし、実現のためには現地の住民や政府機関を含めたステークホルダーの説得や現地研究者の情報技術リテラシー教育など、様々な問題をひとつずつ解決する必要があるため、入念な事前調査が必要である。本年度はネパールのチトワン国立公園を実際に訪れることで、どのような動物がどのような危機にさらされているか事前調査を行った。また、生態音響機材を実際にネパール国内の研究者とともに設置しながら議論を進めることで、生態音響的手法が保全活動に大きく貢献できることを確認した。ネパール国内での調査を福島県浪江町で行っている研究と並行して進めることで、生態音響システムによる生物調査のさらなる発展を促すことができると考えられる。
 
% % \begin{雑誌論文}{1}

% \bibitem{Uesaka2023HCII}
% Uesaka L, Khatiwada AP, Shimotoku D, Parajuli LK, Pandey MR, Kobayashi HH (2023). Applications of Bioacoustics Human Interface System for Wildlife Conservation in Nepal. Accepted for publication in Human Computer Interaction (HCI) International 2023.

% \end{雑誌論文}


% Ishikawa group
\subsection{高速知能システムの研究(石川正俊)}

%  本研究室では、センサ情報処理における並列処理を基盤として、高速・リアルタイム性を有するセンサ情報処理を高度に実現し、高速知能システムとして実装する研究を行っている。具体的には、以下4つの分野での研究開発を行っている。また、各分野での新規産業開拓にも力を注ぎ、研究成果の技術移転、共同研究、事業化等を様々な形で積極的に推進している。

% センサフュージョンの研究では、高速のセンサフィードバックに関する理論並びにシステムアーキテクチャの構築、その高速知能ロボットとしての実装、並びに高速性を活かした新規タスクの実現、特に、センサ情報に基づく人間機械協調システムの開発を行っている。

% ダイナミックビジョンシステムの研究では、高速ビジョンや動的光学系に基づき運動対象の情報を適応的に取得する基礎技術の開発、特に、高速光軸制御や適応光学系の技術開発やトラッキング撮像に関する応用システムの開発を行っている。

% システムビジョンデザインの研究では、高速三次元形状計測や高速質感計測など、並列処理に基づく高速画像処理技術 (理論、アルゴリズム、デバイス) の開発とその応用システムの実現を目指し、特に高速画像処理システムの開発や応用システムの開発を行っている。

% アクティブパーセプションの研究では、実世界における新たな知覚補助技術並びにそれに基づく新しい対話の形の創出を目指し、特に各種高速化技術を用いた能動計測や能動認識を利用した革新的情報環境・ヒューマンインタフェイスの開発を行っている。

% これらの研究では、原則的に並列かつリアルタイムの演算構造を有する現実の物理世界と同等の構造を工学的に実現することを目指しており、そのことにより、現実世界の理解を促すばかりでなく、従来のシステムをはるかに凌駕する性能を有する高速知能システムを生み出すことができ、結果として、まったく新しい情報システムを構築することが可能となる。


%% \begin{雑誌論文}{1}
% \bibitem{01石川正俊01}
% Masatoshi Ishikawa, Idaku Ishii, Hiromasa Oku, Akio Namiki, Yuji Yamakawa, and Tomohiko Hayakawa: Special Issue on High-Speed Vision and its Applications, Journal of Robotics and Mechatronics, Vol.34, No.5, p.911, 2022.

% \bibitem{01石川正俊02}
% Taku Senoo, Atsushi Konno, Yunzhuo Wang, Masahiro Hirano, Norimasa Kishi, and Masatoshi Ishikawa: Tracking of Overlapped Vehicles with Spatio-Temporal Shared Filter for High-Speed Stereo Vision, Journal of Robotics and Mechatronics, Vol.34, No.5, pp.1033-1042, 2022.

% \bibitem{01石川正俊03}
% Hyuno Kim, Yuji Yamakawa, and Masatoshi Ishikawa: Seamless Multiple-Target Tracking Method Across Overlapped Multiple Camera Views Using High-Speed Image Capture, Journal of Robotics and Mechatronics, Vol.34, No.5, pp.1043-1052, 2022.

% \bibitem{01石川正俊04}
% Masatoshi Ishikawa: High-Speed Vision and its Applications Toward High-Speed Intelligent Systems, Journal of Robotics and Mechatronics, Vol.34, No.5, pp.912-935 , 2022.

% \end{雑誌論文}

% \begin{発表}{1}

% \bibitem{01石川正俊05}
% Taku Senoo, Atsushi Konno, Yunzhuo Wang, Masahiro Hirano, Norimasa Kishi, and Masatoshi Ishikawa: Automotive Tracking with High-speed Stereo Vision Based on a Spatiotemporal Shared Filter, the 2022 26th International Conference on System Theory, Control and Computing (ICSTCC2022), Proceedings, pp.613-618, 2022.

% \end{発表}

% \begin{報道}{1}

% \bibitem{01石川正俊06}
% 石川グループ研究室: 世界!オモシロ学者のスゴ動画祭3, NHK, 2022年7月8日, BSプレミアム

% \end{報道}

\subsection{革新的情報環境構築に向けた能動計測型高速トラッキングシステムの研究開発(早川智彦)}

%  2022年度は主に高速画像処理を用いた高速トラッキングシステムの開発を実施し、主に1.白線認識を用いた高速道路における高速撮像システムの研究、2.高速画角切り替えによる画角拡張システムの開発を実施した。全体を通した研究成果として、2件の雑誌論文(査読付)、1件の解説論文、3件の査読付き国際学会予稿と1件の雑誌以外の査読なし国内学会予稿に採択された。

% 1.白線認識を用いた高速道路における高速撮像システムの研究
% 高速道路のトンネルにおける自己位置推定手法として、GPSでは電波を捕捉できないため機能せず、加速度センサでは十分な精度が得られないため、白線認識を利用した自己位置推定手法を開発した。この手法では、白線を画像で撮像することで、その見え方から自己位置を推定することが可能であり、自己位置を元に、トンネル天井面の撮像角度を任意の角度へと補償しながら撮像し続けることが可能となる。これにより、運転手の技量に頼ることなく、狙った位置を撮像し続けることができ、点検の効率向上に寄与する技術といえる。この成果を国際論文誌に投稿し、採択に至った。また、インフラ点検の関連技術をまとめ、関連技術を解説論文にも投稿することで、技術の社会実装が円滑に進むよう努めた。

% 2.高速画角切り替えによる画角拡張システムの開発
% 画角と解像度はトレードオフの関係にあるが、画角をミラーによって高速に切り替えることで、仮想的に画角の拡張を実現する技術を開発した。これにより、例えば高速道路の点検といった応用において、高解像度な画像を少ないカメラ台数で複数回に渡り撮像する状況において、その撮像回数を大幅に減少させることが可能となる。この成果を国際論文誌に投稿し、採択に至った。

%% \begin{招待論文}{1}

% \bibitem{02早川智彦01}
% 早川智彦, 東晋一郎: 時速100km走行での覆工コンクリート高解像度変状検出手法, 建設機械, vol.58, no.4, pp.34-38, 2022.

% \end{招待論文}

% \begin{雑誌論文}{1}

% \bibitem{02早川智彦02}
% Yuriko Ezaki, Yushi Moko, Tomohiko Hayakawa, and Masatoshi Ishikawa: Angle of View Switching Method at High-Speed Using Motion Blur Compensation for Infrastructure Inspection, Journal of Robotics and Mechatronics,Vol. 34, No. 5, pp.985-996, 2022.

% \bibitem{02早川智彦03}
% Tomohiko Hayakawa, Yushi Moko, Kenta Morishita, Yuka Hiruma, Masatoshi Ishikawa: Tunnel Surface Monitoring System with Angle of View Compensation Function based on Self-localization by Lane Detection, Journal of Robotics and Mechatronics, Vol.34, No. 5, pp.997-1010, 2022.

% \end{雑誌論文}

% \begin{査読付}{1}

% \bibitem{02早川智彦04}
% Kairi Mine, Chika Nishimura, Tomohiko Hayakawa, Satoshi Yawata, Dai Watanabe, and Masatoshi Ishikawa: Migration correction technique using spatial information of neuronal images in fiber-inserted mouse under free-running behavior, Conf. on Neural Imaging and Sensing 2023, SPIE Photonics West BiOS/Proc. SPIE, Vol.1236522, pp.1236522: 1-1236522: 5, 2023.

% \bibitem{02早川智彦05}
% Yushan Ke, Yushi Moko, Yuka Hiruma, Tomohiko Hayakawa, Masatoshi Ishikawa: Silk-printed retroreflective markers for infrastructure-maintenance vehicles in tunnels, SPIE Smart Structures and Materials + Nondestructive Evaluation 2022 On Demand (Online), Paper 12046-18, 2022.

% \bibitem{02早川智彦06}
% Yushan Ke, Yushi Moko, Yuka Hiruma, Tomohiko Hayakawa, Elgueta Scarlet, Masatoshi Ishikawa: Silk-printed retroreflective markers for infrastructure-maintenance vehicles in curved tunnels, SPIE Smart Structures and Materials + Nondestructive Evaluation 2023, Paper 12483-40, 2023.

% \end{査読付}

% \begin{発表}{1}

% \bibitem{02早川智彦07}
% 蛭間友香, 早川智彦, 石川正俊: 映像遅延および空間情報を制御可能な手の高速撮像・投影システムの構築, 第27回日本バーチャルリアリティ学会大会(vrsj2022)(札幌, 2022.9.14)/予稿集, 3F5-4, 2022.

% \end{発表}


\subsection{知能ロボットおよび人間機械協調の実現に向けた研究開発(黄守仁)}

%  本年度は主に生産システムの知能化を目指した動的補償ロボットの開発と評価実験、無拘束の視線入力などに基づく人間・ロボットの協調の提案および実装、高速三次元計測によるロボット制御などの研究内容で研究活動を行った。

% 生産システムの知能化を目指した(企業との)共同研究において、前年度の研究を継続し、一般的な産業用ロボットが速度・精度・不確定要素に対する適応能力の三者を同時に成立させることが難しい現状に対して、高帯域でのセンシング・動作によるローカル誤差吸収と低い帯域でのグローバル計画動作を並列に行う動的補償手法を提案し、コンベア上を流れている事前情報(置く位置・姿勢、塗布形状などの情報を指す)のない部品に対する高精度な塗布作業の実現を可能とするロボットを開発した。初期の評価実験で得られた結果を計測自動制御学会システムインテグレーション部門SI2022にて発表を行い、研究成果が認められ、優秀講演賞を受賞した。

% 次に、次世代サイボーグシステムの実現に向けて、無拘束の視線入力とロボットによるセンシング支援・動作支援を統合する人間機械協調手法を提案し、ROS環境での実装および予備実験を行った。室内の日常生活を想定し、人間の視線入力とロボットの視覚センシング支援および動作支援の応用場面を検討した。特に、南デンマーク大学からの博士課程学生が協力研究補助員として短期訪問をしている間、システム実装に向けてとても有意義な交流を行った。

% また、企業との共同研究課題として、高速三次元計測を動的補償ロボット制御に統合するシステムの設計およびシミュレーションモデル構築など、前年度の研究活動も継続して取り組んでいる。

%% \begin{受賞}{1}

% \bibitem{03黄守仁01}
% 黄守仁,村上健一, 石川正俊: 対象の事前情報必要としない動的塗布応用に向けたロボットの実現, 第23回計測自動制御学会システムインテグレーション部門講演会(SI2022), 講演会論文集, pp.999-1001, SI2022 優秀講演賞, 2022.

% \end{受賞}

% \begin{雑誌論文}{1}

% \bibitem{03黄守仁02}
% Kenichi Murakami, Shouren Huang, Masatoshi Ishikawa, and Yuji Yamakawa: Fully Automated Bead Art Assembly for Smart Manufacturing Using Dynamic Compensation Approach, J. Robot. Mechatron., Vol.34, No.5, pp.936-945, 2022. 

% \bibitem{03黄守仁03}
% Shouren Huang, Kenichi Murakami, Masatoshi Ishikawa, Yuji Yamakawa: Robotic Assistance Realizing Peg-and-Hole Alignment by Mimicking the Process of an Annular Solar Eclipse, Journal of Robotics and Mechatronics, Vol.34 No.5, pp.946-955, 2022. 

% \end{雑誌論文}

% \begin{発表}{1}

% \bibitem{03黄守仁04}
% 黄守仁, 村上健一, 石川正俊: 対象の事前情報必要としない動的塗布応用に向けたロボットの実現, 第23回計測自動制御学会システムインテグレーション部門講演会(SI2022), 講演会論文集, pp.999-1001, 2022.

% \bibitem{03黄守仁05}
% 村上健一,黄守仁,石川正俊,山川雄司: 動的補償を用いたビーズピッキング, 第40回日本ロボット学会学術講演会 (RSJ2022), 予稿集, 4C1-05, 2022.

% \bibitem{03黄守仁06}
% Shouren Huang, Yongpeng Cao, Kenichi Murakami, Masatoshi Ishikawa,Yuji Yamakawa: Bimanual Coordination Protocol for the Inter-Limb Transmission of Force Feedback, 第40回日本ロボット学会学術講演会 (RSJ2022), 予稿集, 2C1-05, 2022.

% \end{発表}


\subsection{ダイナミックビジョンシステムの研究開発(末石智大)}

%  高速画像処理および高速光学系制御を用いた、動的検査技術とヒューマンインターフェースに関するダイナミックビジョンシステムの研究を実施した。

% 動的検査技術は、実世界の動的かつ複雑な現象を適応的にデータ化し、意味のある形で活用する技術となるものであり、昨年度に引き続き本年度もマイクロサッカードと呼ばれる眼球微振動などを対象として実施した。
% 被験者を拘束する負荷や時間効率の観点から、動的状態への検査技術の発展の期待は大きいと考えられる。頭部固定を必要とせずリラックスした状態の人間のマイクロサッカード検出に向けて、回転ミラーや液体可変焦点レンズなどの光学素子を高速に制御しつつ高解像度合焦画像計測を達成することで、ダイナミックビジョンシステムにおける動的検査への基礎技術の研鑽を進めている。
% 継続して開発を進めているマイクロサッカード計測システムの改良に加え、定量的評価のための動的眼球模型の両眼化を含む開発・改良、照明制御を伴う視線方向の高精度計測戦略、マイクロサッカード検知アルゴリズムを含む統合システム開発などの成果を実現した。

% 本年度は眼球運動以外にも、卓球ボールの回転運動をリアルタイムに計測するシステムや、液体可変焦点レンズを含むカメラ系の効率的な構成手法などにも取り組んだ。
% 特に球技スポーツであるテニスのライン判定を目的とした高速ビジョンと落下位置予測技術の融合システムの初期検討も実施し、本内容で国内学会において優秀講演賞を受賞した。
% ヒューマンインターフェースに関しては、ベクター図形を描画可能なレーザー投影システムに不可視のバイナリ符号情報を埋め込んだ、自転車競技などへの活用が期待される、高速自己位置推定システムを開発し論文誌に掲載された。
% 光学系・照明系の適切な同期・校正手法を新たに開発することで、円に限定されない非対称なベクター図形を用いた情報提示と高速トラッキングの両立を可能とし、スポーツにおける複雑な身体運動のデータ化や即時フィードバックによる効率的な運動学習が期待される。

% 本年度は総じて、ヒトの眼球やボールの跳躍位置など瞬間的な動作に着目した高速センシングに関する新たな計測制御技術を、特に実世界応用へと繋がる成果として創出した。

%\begin{招待講演}{1}

\bibitem{02末石智大01}
末石智大: 高速光学系制御による動的イメージングシステムとその応用, 高速度イメージングとフォトニクスに関する総合シンポジウム2023, 近畿大学, 大阪, 2023.12

\end{招待講演}

\begin{受賞}{1}
\bibitem{02末石智大02}
Himari Tochioka, Tomohiro Sueishi, and Masatoshi Ishikawa: SICE Annual Conference International Award, SICE Annual Conference 2023,2023.9

\bibitem{02末石智大03}
田畑智志,末石智大,宮下令央,石川正俊: 2023年 計測自動制御学会 システムインテグレーション部門SI2023 優秀講演賞,第24回計測自動制御学会システムインテグレーション部門講演会 (SI2023), 新潟,2023.12

\end{受賞}

\begin{査読付}{1}
\bibitem{02末石智大04}
Himari Tochioka, Tomohiro Sueishi, and Masatoshi Ishikawa: Bounce Mark Visualization System for Ball Sports Judgement Using High-Speed Drop Location Prediction and Preceding Mirror Control, SICE Annual Conference 2023 (SICE2023),2023.9

\bibitem{02末石智大05}
横山恵子,井上満晶,末石智大,谷内田尚司,石川正俊: 非接触・非拘束型の眼球撮影装置を用いた眼球微細運動検知システム, ビジョン技術の実利用ワークショップ (ViEW2023), pp.413-418, IS3-8,横浜,2023.12

\end{査読付}

\begin{発表}{1}
\bibitem{02末石智大06}
栃岡陽麻里,末石智大,石川正俊: ボールのバウンド位置予測に基づくダイナミックプロジェクションマッピングの開発, 第28回日本バーチャルリアリティ学会大会 (VRSJ2023), 論文集,3D2-01, 八王子, 2023.9

\bibitem{02末石智大07}
田畑智志,末石智大,宮下令央,石川正俊: 円筒位置姿勢情報の高速フィードバックを用いたダイナミックアナモルフォーシスシステムの開発, 第24回計測自動制御学会システムインテグレーション部門講演会 (SI2023), 講演会論文集, pp.1280-1285, 新潟, 2023.12

\end{発表}


\subsection{高速3次元形状計測技術の評価と応用(宮下令央)}

%  本年度は昨年度開発した高速3次元形状計測技術の評価、および関連する高速画像処理システムの研究・発表を行った。
%  新たに開発した高速3次元形状計測技術によって、従来手法よりも高解像度かつ高精度な3次元形状計測が1,000fpsを超える高速性と低遅延性を維持したまま実現できることを確認し、成果を国内外で発表した。

% また、高速3次元形状計測の関連研究として、従来のテレセントリック光学系による画像検査では困難な長尺の棒材の寸法検測を非テレセントリック光学系によって実現するシステムや、高速3次元形状計測と高速法線計測を統合して高密度かつ高精度かつ高速な3次元形状計測を実現するシステム、さらにイメージセンサに高速画像処理を行うプロセッサを並置したビジョンチップを用いて小型化を実現した高速光軸制御システムについて、論文誌において発表を行った。

% さらに、人間の錯覚を利用して仮想的な運動を付加するダイナミックプロジェクションマッピング技術をコンピュータグラフィックスに応用し、アニメーションのレンダリングを高速化する研究を行い、被験者実験を通して得た人間の知覚特性に関する知見について国際学会において発表を行った。
% 今後は高速3次元形状計測技術を発展させ、質感計測やダイナミックプロジェクションマッピングへの応用を進めていく予定である。

% また、解説論文の執筆、研究会の運営や査読を通して学会への貢献を続けている。

%% \begin{招待論文}{1}

% \bibitem{05宮下令央01}
% 宮下 令央, 末石 智大, 田畑 智志, 早川 智彦, 石川 正俊: 高速ビジョンが拓くダイナミックプロジェクションマッピング技術, 電子情報通信学会 通信ソサイエティマガジン B-plus, 解説論文, No.64, pp.275-284, 2023. 

% \end{招待論文}

% \begin{雑誌論文}{1}

% \bibitem{05宮下令央02}
% Leo Miyashita, Masatoshi Ishikawa: Portable High-speed Optical Gaze Controller with Vision Chip, Journal of Robotics and Mechatronics(JRM), Vol.34, No.5, pp.1133-1140, 2022.

% \bibitem{05宮下令央03}
% Leo Miyashita, Yohta Kimura, Satoshi Tabata, Masatoshi Ishikawa: High-speed Depth-normal Measurement and Fusion Based on Multiband Sensing and Block Parallelization, Journal of Robotics and Mechatronics (JRM), Vol.34, No.5, pp.1111-1121, 2022.

% \bibitem{05宮下令央04}
% Leo Miyashita, Masatoshi Ishikawa: Real-Time Inspection of Rod Straightness and Appearance by Non-Telecentric Camera Array, Journal of Robotics and Mechatronics (JRM), Vol.34, No.5, pp.975-984, 2022.

% \bibitem{05宮下令央05}
% Yunpu Hu, Leo Miyashita, and Masatoshi Ishikawa: Differential Frequency Heterodyne Time-of-Flight Imaging for Instantaneous Depth and Velocity Estimation. ACM Trans. Graph. Vol.42, No.1, Article 9, pp.9, 2022.

% \end{雑誌論文}

% \begin{査読付}{1}

% \bibitem{05宮下令央06}
% 宮下 令央, 田畑 智志, 石川 正俊: パラレルバスパターンによる高速低遅延3次元形状計測, 計測自動制御学会, 第39回 センシングフォーラム, 1B1-1, 予稿集 pp.49-54, 2022.

% \bibitem{05宮下令央07}
% Leo Miyashita, Satoshi Tabata, Masatoshi Ishikawa: High-speed and Low-latency 3D Sensing with a Parallel-bus Pattern, International Conference on 3D Vision(3DV2022), 2022.

% \bibitem{05宮下令央08}
% Leo Miyashita, Kentaro Fukamizu, Yuki Kubota, Tomohiko Hayakawa, Masatoshi Ishikawa: Real-time animation display based on optical illusion by overlaid luminance changes, SPIE Optical Architectures for Displays and Sensing in Augmented, Virtual, and Mixed Reality(AR, VR, MR)IV, Oral, paper 12449-8, 2023.

% \end{査読付}


% \subsection{小型高速三次元スキャナの開発および可変光学系による技術拡張に関する研究(田畑智志)}

%  本年度は主に小型高速三次元スキャナの開発を昨年度に引き続き実施するとともに、計測システムの可変焦点化による計測技術の向上方法の検討と、形状計測の高精度高解像化やダイナミックプロジェクションマッピング技術の広範囲化に関する実験を実施した。

% 高速三次元スキャン技術は、運動する物体の形状取得やロボットにおける外界認識・インタラクションなど、高速性・リアルタイム性が要求される応用において重要である。昨年度に引き続き、小型ユニットを用いた1,000fpsでの高速小型三次元スキャナの開発を行い、国内学会での口頭発表や招待論文、報道による周知を行った。
% リアルタイムのモデル生成と運動補正を組みあわせることで1,000fpsの高速性を維持したまま、1辺10cmの立方体に収まるサイズで片手に持ってスキャン可能な計測システムを実現している。また、この技術をさらに発展させるため、さらなる統合アプローチの検討を進めている。

% さらに、三次元形状計測について、計測距離範囲・精度を向上させるため高速可変焦点レンズの導入を進め予備実験を実施した。また、位相シフト法における計測精度の向上や高解像化に際して課題となる点を精査し、それぞれの課題を解決するためのアルゴリズムの開発を進めている。

% また、ダイナミックプロジェクションマッピングに関しては、対象の三次元的な情報を計測・利用するシステムに対する回転ミラーを用いたトラッキングアプローチの拡張を進めており、実際の応用を通じた実験を行っている。



%% \begin{招待論文}{1}

% \bibitem{06田畑智志01}
% 田畑智志, 渡辺義浩, 石川正俊: 小型高速三次元スキャナの研究開発と将来展望, 日本ロボット工業会機関紙『ロボット』, 271号, pp.45-47, 2023.

% \end{招待論文}

% \begin{雑誌論文}{1}

% \bibitem{06田畑智志02}
% Lihui Wang, Satoshi Tabata, Hongjin Xu, Yunpu Hu, Yoshihiro Watanabe, and Masatoshi Ishikawa: Dynamic depth-of-field projection mapping method based on a variable focus lens and visual feedback, Optics Express, Vol.31, Issue 3, pp.3945-3953, 2023.

% \bibitem{06田畑智志03}
% Hao Xu, Satoshi Tabata, Haowen Liang, Lihui Wang, and Masatoshi Ishikawa: Accurate measurement of virtual image distance for near-eye displays based on auto-focusing, Applied Optics, Vol.61, Issue 30, pp.9093-9098, 2022.

% \bibitem{06田畑智志04}
% Yuping Wang, Senwei Xie, Lihui Wang, Hongjin Xu, Satoshi Tabata, and Masatoshi Ishikawa, ARSlice: Head-Mounted Display Augmented with Dynamic Tracking and Projection, Journal of Computer Science and Technology, Vol.37, Issue 3, pp.666-679, 2022.

% \end{雑誌論文}

% \begin{査読付}{1}

% \bibitem{06田畑智志05}
% Yuping Wang, Senwei Xie, Lihui Wang, Hongjin Xu, Satoshi Tabata, Masatoshi Ishikawa: Head-Mounted Display Augmented with Dynamic Tracking and Projection, The 10th international conference on Computational Visual Media(CVM 2022), 2022.

% \end{査読付}

% \begin{発表}{1}

% \bibitem{06田畑智志06}
% 田畑智志, 渡辺義浩, 石川正俊: 小型高速三次元スキャナの開発, 第40回日本ロボット学会学術講演会(RSJ2022), 予稿集, 2B1-07, 2022.

% \end{発表}

% \begin{報道}{1}

% \bibitem{06田畑智志07}
% 田畑智志: 毎秒1000回撮像で立体計測 小型高速3Dスキャナー ロボアームの高速操作向け 東大など開発, 日刊工業新聞, 2022年9月27日.

% \end{報道}



\section{データ科学研究部門 成果要覧}
\begin{招待講演}{1}

\bibitem{01早川智彦01}
Tomohiko Hayakawa:  User Performance based on the Effects of Low Video Latency between Visual and Haptic Information in Immersive Environments (Invited), The 6th International Conference on Intelligent Robotics and Control Engineering (IRCE 2023), 2023.8

\bibitem{01早川智彦02}
Tomohiko Hayakawa:  Optical axis control methods for infrastructure inspection, 2nd Intl. Conference Advances in 3OM , OPT23-42,Timisoara, 2023.12


\bibitem{02末石智大01}
末石智大: 高速光学系制御による動的イメージングシステムとその応用, 高速度イメージングとフォトニクスに関する総合シンポジウム2023, 近畿大学, 大阪, 2023.12


\bibitem{hanai-simpo}
華井雅俊、
"ARIM-mdxデータシステム",
ARIM「第2回革新的なエネルギー変換を可能とするマテリアル領域」シンポジウム,
2024年1月

\bibitem{hanai-utac}
華井雅俊,
"ARIM-mdxデータシステム: 材料実験データの利活用に向けた実験施設のDX化", 
第一回UDAC-SRIS合同勉強会,
2023年12月

\bibitem{hanai-gizyutsu}
華井雅俊、
"ARIM-mdxデータシステム"
ARIM 第1回計測技術スタッフ全体研修会,
2023年12月

\bibitem{kawamura_yitp}
河村光晶,
``超伝導密度汎関数理論の精度検証と応用'',
超伝導研究の発展と広がり,
京都大学基礎物理学研究所,
2023年12月.
\bibitem{ykuga45871732}
空閑洋平, データセンタハードウェアへのソフトウェア脆弱試験の適応, Society 5.0時代の安心・安全・信頼を支える基盤ソフトウェア技術の構築, 2024年3月.


% \bibitem{fugaku} 鈴村豊太郎「夢の形~未来のコンピュータ~」(パネリスト)、スーパーコンピュータ「富岳」第2回成果創出加速プログラムシンポジウム「富岳百景」、2022年12月21日
% \bibitem{jsse1}鈴村豊太郎,「データ活用社会創成プラットフォームmdxおよび 大規模グラフニューラルネットワーク」, 第34回CCSEワークショップ「原子力材料研究開発におけるDX推進の現状と将来:原子力材料研究開発の革新と新展開」, 2023年2月24日
% \bibitem{jsse2}鈴村豊太郎,「人工知能の最先端研究に迫る ~大規模グラフニューラルネットワークの世界へ」, 東京大学柏キャンパス一般公開2022、特別講演会、2022年10月22日
% \bibitem{rakuten}Toyotaro Suzumura, “How Will Data and AI Change the World?”,  Rakuten Optimism Conference, Tokyo, Japan, September 29, 2022
% \bibitem{rccs}Toyotaro Suzumura,  “Large-Scale Graph Neural Networks for Real-World Industrial Applications”, The 5th R-CCS International Symposium, Kobe Japan, February 7, 2023
% \bibitem{france}Toyotaro Suzumura,  “Large-Scale Graph Neural Networks for Real-World Industrial Applications”, International Workshop “HPC challenges for new extreme scale application” held by French Alternative Energies and Atomic Energy Commission, Paris, France, March 6, 2023
% \bibitem{barcelona}Toyotaro Suzumura,  “Large-Scale Graph Neural Networks for Real-World Industrial Applications”, Barcelona Supercomputing Center, Barcelona, Spain, March 10, 2023

% sample
% \bibitem{sample-kobayashi3-1}
% Hill Hiroki Kobayashi, mdx: A Cloud Platform for Supporting Data Science and Cross-Disciplinary Research Collaborations, the Nepal JSPS Alumni Association (NJAA), hosted its 7th Symposium, 29 November, 2022.

\bibitem{ireneli-1}
Irene Li, A Journey from Transformers to Large Language Models: an Educational Perspective, 2023 the 1st International Conference on AI-generated Content (AIGC2023), Aug, 2023


\end{招待講演}

\begin{招待論文}{1}

\bibitem{kawamura_kotai}
吉見一慶, 本山裕一, 青山龍美, 川島直輝, 河村光晶,
物性研の計算物性科学コミュニティ支援活動,
固体物理 Vol. 58, No. 9, 29 (2023).
\bibitem{ykuga45871752}
河原大輔, 空閑洋平, 黒橋禎夫, 鈴木潤, 宮尾祐介, LLM-jp: 日本語に強い大規模言語モデルの研究開発を行う組織横断プロジェクト, 自然言語処理, 31, 1, pp266-279, 2024年.





\end{招待論文}

\begin{受賞}{1}

\bibitem{02末石智大02}
Himari Tochioka, Tomohiro Sueishi, and Masatoshi Ishikawa: SICE Annual Conference International Award, SICE Annual Conference 2023,2023.9

\bibitem{02末石智大03}
田畑智志,末石智大,宮下令央,石川正俊: 2023年 計測自動制御学会 システムインテグレーション部門SI2023 優秀講演賞,第24回計測自動制御学会システムインテグレーション部門講演会 (SI2023), 新潟,2023.12

\bibitem{kawamura_hpci1}
HPCIソフトウェア賞【普及部門賞】最優秀賞, 
「計算物質科学ソフトウェア普及活動」, 
MateriAppsチーム (井戸 康太、福田 将大、笠松 秀輔、三澤 貴宏) および PASUMSチーム (本山 裕一, 河村 光晶, 吉見 一慶), 2
023年5月.

\bibitem{kawamura_hpci2}
HPCIソフトウェア賞【開発部門賞】最優秀賞,
「mVMC」, 
mVMC開発チーム (井戸 康太, 森田 悟史, 吉見 一慶, 本山 裕一, 加藤 岳生, 河村 光晶, Ruquing Xu, 今田 正俊, 三澤 貴宏), 
2023年5月.

\bibitem{kawamura_hpci3}
HPCIソフトウェア賞【開発部門賞】優秀賞, 
「$\mathcal{H}\Phi$」, 
$\mathcal{H}\Phi$開発チーム (河村 光晶, 吉見 一慶, 三澤 貴宏, 井戸 康太, 本山 裕一, 山地 洋平), 
2023年5月.
\bibitem{ykuga43010880}
空閑洋平, 山下記念研究賞, 情報処理学会(OS研究会), 2024年3月.

\bibitem{ykuga43010877}
空閑洋平, 山下記念研究賞, 情報処理学会(IOT研究会), 2024年3月.

\bibitem{ykuga43010874}
空閑洋平, 中村遼, 藤村記念ベストプラクティス賞, 情報処理学会(IOT研究会), 2023年7月.

\bibitem{ykuga9999}
中村遼, 空閑洋平, 明石邦夫, Most Innovative for HPC Uses Award, Data Mover Challenge 2023, 2024年2月.



\bibitem{ireneli-2}
Boming Yang, Dairui Liu, Toyotaro Suzumura, Ruihai Dong and Irene Li,\lq\lq Going Beyond Local: Global Graph-Enhanced Personalized News Recommendations", Proceedings of the 17th ACM Conference on Recommender Systems  (RecSys 2023), 2023 (Best Student Paper Award)

\end{受賞}

\begin{著書}{1}



% \bibitem{fl-book}
% Toyotaro Suzumura, Yi Zhou, Ryo Kawahara, Nathalie Baracaldo, Heiko Ludwig,
% "Federated Learning for Collaborative Financial Crimes Detection",
% Ludwig, H., Baracaldo, N. (eds) Federated Learning. Springer, Cham. https://doi.org/10.1007/978-3-030-96896-0\_20

% \bibitem{jsai-gnn}
% 鈴村 豊太郎, 金刺 宏樹, 華井 雅俊, グラフニューラルネットワークの広がる活用分野, 人工知能, 2023, 38 巻, 2 号, p. 139-148, 公開日 2023/03/02, Online ISSN 2435-8614, Print ISSN 2188-2266, https://doi.org/10.11517/jjsai.38.2\_139


\end{著書}

\begin{雑誌論文}{1}


% \bibitem{gnn-glass}
% Hayato Shiba, Masatoshi Hanai, Toyotaro Suzumura, and Takashi Shimokawabe, "BOTAN: BOnd TArgeting Network for prediction of slow glassy dynamics by machine learning relative motion." The Journal of Chemical Physics 158, no. 8, 084503, 2022.
\bibitem{kawamura_la2ios2}
H. Ishikawa, T. Yajima, D. Nishio-Hamane, S. Imajo, K. Kindo, and M. Kawamura,
``Superconductivity at 12 K in ${\mathrm{La}}_{2}{\mathrm{IOs}}_{2}$: A $5d$ metal with osmium honeycomb layer'',
Physical Review Materials \textbf{7}, 054804 (2023).

\bibitem{kawamura_hphi}
K. Ido, M. Kawamura, Y. Motoyama, K. Yoshimi, Y. Yamaji, S. Todo, N. Kawashima, T. Misawa,
``Update of $\mathcal{H}\Phi$: Newly added functions and methods in versions 2 and 3'',
Computer Physics Communications \textbf{298}, 109093 (2024).

\bibitem{kawamura_mos2}
F. Ozaki, Dr. S. Tanaka, Y. Choi, W. Osada, K. Mukai, M. Kawamura, M. Fukuda, M. Horio, T. Koitaya, S. Yamamoto, I. Matsuda, T. Ozaki, and J. Yoshinobu,
``Hydrogen-induced Sulfur Vacancies on the $\mathrm{MoS}_2$ Basal Plane Studied by Ambient Pressure XPS and DFT Calculations'',
ChemPhysChem e202300477 (2023).

\bibitem{kawamura_max}
M. Khazaei, S. Bae, R. Khaledialidusti, A. Ranjbar, H. Komsa, S. Khazaei, M. Bagheri, V. Wang, Y. Mochizuki, M. Kawamura, G. Cuniberti, S. M. V. Allaei, K. Ohno, H. Hosono, and H. Raebiger, 
``Superlattice MAX Phases with A-Layers Reconstructed into 0D-Clusters, 1D-Chains, and 2D-Lattices'',
The Journal of Physical Chemistry C \textbf{127}, 30, 14906 (2023).

 \bibitem{kobayashi1-1}
 Wenjing Li, Xiaodan Shi, Dou Huang, Xudong Shen, Jinyu Chen, Hill Hiroki Kobayashi, Haoran Zhang, Xuan Song, Ryosuke Shibasaki,  "PredLife: Predicting Fine-Grained Future Activity Patterns ", IEEE Transactions on Big Data 9(6) 1658-1669.

 \bibitem{kobayashi1-2}
 Wenjing Li, Haoran Zhang, Jinyu Chen, Peiran Li, Yuhao Yao, Xiaodan Shi, Mariko Shibasaki, Hill Hiroki Kobayashi, Xuan Song 0001, Ryosuke Shibasaki,  "Metagraph-Based Life Pattern Clustering With Big Human Mobility Data", IEEE Trans. Big Data 9(1) 227-240.
 
\bibitem{ykuga45871761}
空閑洋平, 中村遼, 遠隔会議システムの計測データを用いたネットワーク品質計測, 情報処理学会論文誌, 65, 3, pp646-655, 2024年3月.

\bibitem{dmiya1}
Masayuki Jimichi, Yoshinori Kawasaki, Daisuke Miyamoto, Chika Saka, Shuichi Nagata,
"Double-Log Modeling of Financial Data with Skew-Symmetric Error Distributions from Viewpoints of Exploratory Data Analysis and Reproducible Research",
Symmetry, MPDI, 15(9), 19 pages, September 2023. 

\bibitem{dmiya2}
Shun Yonamine, Yuzo Taenaka, Youki Kadobayashi, Daisuke Miyamoto,
"Design and implementation of a sandbox for facilitating and automating IoT malware analysis with techniques to elicit malicious behavior: case studies of functionalities for dissecting IoT malware",
Journal of Computer Virology and Hacking Techniques, Springer, 19(2), pp.149-163, March 2023.


% \bibitem{feta}
% Bastos, Anson, Abhishek Nadgeri, Kuldeep Singh, Hiroki Kanezashi, Toyotaro Suzumura, and Isaiah Onando Mulang, "How Expressive Are Transformers in Spectral Domain for Graphs?", Transactions on Machine Learning Research (TMLR) ISSN 2835-8856, Journal of Machine Learning Research, 2022.

% \bibitem{sample}
% Irene Li, Jessica Pan, Jeremy Goldwasser, Neha Verma, Wai Pan Wong, Muhammed Yavuz Nuzumlalı, Benjamin Rosand, Yixin Li, Matthew Zhang, David Chang, R. Andrew Taylor, Harlan M. Krumholz, Dragomir Radev,\lq\lq Neural Natural Language Processing for unstructured data in electronic health records: A review", Computer Science Review, volume 46, 2022.

% \end{雑誌論文}
% \begin{雑誌論文}{1}  % journals

% % sample
% % \bibitem{sample-kobayashi1-3}
% % Wenjing Li, Haoran Zhang, Jinyu Chen, Peiran Li, Yuhao Yao, Xiaodan Shi,  Mariko Shibasaki, Hill Hiroki Kobayashi, Xuan Song and Ryosuke Shibasaki, \lq\lq Metagraph-Based Life Pattern Clustering With Big Human Mobility Data", IEEE Transactions on Big Data, Feb, 2023.

\end{雑誌論文}

\begin{査読付}{1}


\bibitem{01早川智彦03}
Yushi Moko, Yuka Hiruma, Tomohiko Hayakawak, Yuriko Ezaki, Yoshimasa Onishi, Masatoshi Ishikawa:A method for calculating crack width using chalking marks in low-contrast 2D images acquired during high-speed driving, In Proc. Of SPIE 12952, 4,2024.3

\bibitem{01早川智彦04}
Tomohiko Hayakawak, Yuka Hiruma, Ke Yushan, Masatoshi Ishikawa
Label-free position tracking in stuffed sparrow using phosphorescence, Conf. on Label-free Biomedical Imaging and Sensing (LBIS) 2024, SPIE Photonics West BiOS, San Francisco,2024.1

\bibitem{01早川智彦05}
Yuki Kubota, Tomohiko Hayakawa, and Masatoshi Ishikawa: Visual Search Tasks under Spatio-Temporal Control Synchronized with Eye Movements, In Proceedings of the 2023 Symposium on Eye Tracking Research and Applications (ETRA '23), Association for Computing Machinery, Article 43, pp.1-2,New York, USA, 2023.6

\bibitem{01早川智彦06}
Yushi Moko, Yuka Hiruma, Tomohiko Hayakawak, Yoshimasa Onishi, Masatoshi Ishikawa:  High-Speed Localization Estimation Method Using Lighting Recognition in Tunnels, 2023 7th International Conference on Intelligent Traffic and Transportation(ICITT 2023) , ML755,Madrid,2023.9

\bibitem{01早川智彦07}
Tomohiko Hayakawa, Yuka Hiruma, Yushan Ke, Masatoshi Ishikawa: Active thermal marker using thermal images of heated areas with visible semiconductor laser, the 10th edition of the International Conference on Optical and Photonic Engineering(icOPEN 2023), 15617, Singapore, 2023.11

\bibitem{02末石智大04}
Himari Tochioka, Tomohiro Sueishi, and Masatoshi Ishikawa: Bounce Mark Visualization System for Ball Sports Judgement Using High-Speed Drop Location Prediction and Preceding Mirror Control, SICE Annual Conference 2023 (SICE2023),2023.9

\bibitem{02末石智大05}
横山恵子,井上満晶,末石智大,谷内田尚司,石川正俊: 非接触・非拘束型の眼球撮影装置を用いた眼球微細運動検知システム, ビジョン技術の実利用ワークショップ (ViEW2023), pp.413-418, IS3-8,横浜,2023.12


\bibitem{hanai-UCC}
Masatoshi Hanai, Mitsuaki Kawamura, Ryo Ishikawa, Toyotaro Suzumura, Kenjiro Taura
"Cloud Data Acquisition from Shared-Use Facilities in A University-Scale Laboratory Information Management System."
In Proceedings of the 16th IEEE/ACM International Conference on Utility and Cloud Computing (UCC 2023), December 5, 2023, Taormina.

\bibitem{hanai-xiaohang}
Xiaohang Xu, Toyotaro Suzumura, Jiawei Yong, Masatoshi Hanai, Chuang Yang, Hiroki Kanezashi, Renhe Jiang, Shintaro Fukushima, 
"Revisiting Mobility Modeling with Graph: A Graph Transformer Model for Next Point-of-Interest Recommendation."
In Proceedings of the 31st ACM International Conference on Advances in Geographic Information Systems (SIGSPATIAL ’23). Association for Computing Machinery, New York, NY, USA, Article
94, 1–10.

\bibitem{job-jsai}
脇聡志, 鈴村豊太郎, 金刺宏樹, 華井雅俊, 小林秀. "強化学習によるマッチング数を最大化するジョブ推薦システム." 人工知能学会全国大会論文集 第 37 回 (2023), 一般社団法人 人工知能学会, 2023.

 \bibitem{kobayashi2-1}
 Usman Haider, Muhammad Hanif, Hill Hiroki Kobayashi, Laxmi Kumar Parajuli, Daisuké Shimotoku, Ahmar Rashid, Sonia Safeer, "Bioacoustics Signal Classification Using Hybrid Feature Space with Machine Learning", Proceedings of 15th International Conference on Computer and Automation Engineering(ICCAE), 2023.  

 \bibitem{kobayashi2-2}
 Leo Uesaka, Ambika Prasad Khatiwada, Daisuk\'e Shimotoku, Laxmi Kumar Parajuli, Manish Raj Pandey, Hill Hiroki Kobayashi, "Applications of Bioacoustics Human Interface System for Wildlife Conservation in Nepal", Proceedings of 2023 International Conference on Human-Computer Interaction (HCII 2023), 2023.  

\bibitem{ykuga43404131}
Ryo Nakamura, Yohei Kuga, Multi-threaded scp: Easy and Fast File Transfer over SSH, Practice and Experience in Advanced Research Computing, 2023年7月.



\bibitem{mobgt}
Xiaohang Xu, Toyotaro Suzumura, Jiawei Yong, Masatoshi Hanai, Chuang Yang, Hiroki Kanezashi, Renhe Jiang, and Shintaro Fukushima. 2023. Revisiting Mobility Modeling with Graph: A Graph Transformer Model for Next Point-of-Interest Recommendation. In Proceedings of the 31st ACM International Conference on Advances in Geographic Information Systems (SIGSPATIAL '23). Association for Computing Machinery, New York, NY, USA, Article 94, 1–10.

\bibitem{glory}
Boming Yang, Dairui Liu, Toyotaro Suzumura, Ruihai Dong, and Irene Li. 2023. Going Beyond Local: Global Graph-Enhanced Personalized News Recommendations. In Proceedings of the 17th ACM Conference on Recommender Systems (RecSys '23). Association for Computing Machinery, New York, NY, USA, 24–34.

\bibitem{sigir}
Md Mostafizur Rahman, Daisuke Kikuta, Satyen Abrol, Yu Hirate, Toyotaro Suzumura, Pablo Loyola, Takuma Ebisu and Manoj Kondapaka, "Exploring 360-Degree View of Customers for Lookalike Modeling",  SIGIR'23 (The 46th International ACM
SIGIR Conference on Research and Development in Information
Retrieval) 

\bibitem{kg-kp}
Anson Bastos, Kuldeep Singh, Abhishek Nadgeri, Johannes Hoffart, Toyotaro Suzumura, Manish Singh,
"Can Persistent Homology provide an efficient alternative for Evaluation of Knowledge Graph Completion Methods?"
In proceedings of The Web Conference (WWW), 2023.

\bibitem{job-aaai}
Waki, Satoshi, Toyotaro Suzumura, and Hiroki Kanezashi. "Optimizing Matching Markets with Graph Neural Networks and Reinforcement Learning." In Workshop on Recommendation Ecosystems: Modeling, Optimization and Incentive Design, AAAI 2024.

\bibitem{num-aaai}
Putra, Refaldi, and Toyotaro Suzumura. "On the Role of Numerical Encoding in Foundation Model of Sequential Recommendation with Sequential Indexing." In Workshop on Recommendation Ecosystems: Modeling, Optimization and Incentive Design. 2024, AAAI 2024


\bibitem{job-jsai}
脇聡志, 鈴村豊太郎, 金刺宏樹, 華井雅俊, 小林秀. "強化学習によるマッチング数を最大化するジョブ推薦システム." 人工知能学会全国大会論文集 第 37 回 (2023), 一般社団法人 人工知能学会, 2023.

% \bibitem{feng2022diffuser}
% Aosong Feng and Irene Li and Yuang Jiang andRex  Ying,\lq\lq Diffuser: Efficient Transformers with Multi-hop Attention Diffusion for Long Sequences", Proceedings of Thirty-Seventh AAAI Conference on Artificial Intelligence (AAAI), 2023.

% \end{査読付}
\begin{査読付}{1}  % papers (peer-reviewed)

\bibitem{ireneli-3}
Irene Li, Thomas George, Alexander Fabbri, Tammy Liao, Benjamin Chen, Rina Kawamura, Richard Zhou, Vanessa Yan, Swapnil Hingmire and Dragomir Radev, \lq\lq A Transfer Learning Pipeline for Educational Resource Discovery with Application in Leading Paragraph Generation", BEA workshop at Association for Computational Linguistics, 2023

\bibitem{ireneli-4}
Irene Li and Boming Yang, \lq\lq NNKGC: Improving Knowledge Graph Completion with Node Neighborhoods", DL4KG Workshop at International Semantic Web Conference, 2023

\bibitem{ireneli-5}
Linxin Song, Jieyu Zhang, Lechao Cheng, Pengyuan Zhou, Tianyi Zhou and Irene Li, \lq\lq NLPBench: Evaluating Large Language Models on Solving NLP Problems", Instruction Workshop at Conference on Neural Information Processing Systems, 2023


\bibitem{ireneli-6}
Linxin Song, Yan Cui, Ao Luo, Freddy Lecue and Irene Li, \lq\lq Better Explain Transformers by Illuminating Important Information", (Findings) The European Chapter of the Association for Computational Linguistics, 2024

% sample
% \bibitem{sample-kobayashi2-1}
% Daisuk\'e Shimotoku, Tian Yuan, Laxmi Kumar Parajuli and Hill Hiroki Kobayashi,\lq\lq Participatory Sensing Platform Concept for Wildlife Animals in the Himalaya Region, Nepal", Proceedings of 2022 International Conference on Human-Computer Interaction (HCII 2022), 2022.  

\end{査読付}
% \begin{査読付}{3}

% \bibitem{xsig-limin-hanai}
% Limin Wang, Masatoshi Hanai, Toyotaro Suzumura, Shun Takashige, Kenjiro Taura, "On Data Imbalance in Molecular Property Prediction with Pre-training" xSIG 2023 (submitted)

% \bibitem{xsig-takashige-hanai}
% Shun Takashige, Masatoshi Hanai, Toyotaro Suzumura, Limin Wang, Kenjiro Taura, "Is Self-Supervised Pretraining Good for Extrapolation in Molecular Property Prediction?" xSIG 2023 (submitted)

% \bibitem{stgtrans-xiaohang}
% Xiaohang Xu, Toyotaro Suzumura, Jiawei Yong, Masatoshi Hanai, Chuang Yang, Hiroki Kanezashi, Renhe Jiang, Shintaro Fukushima, "Spatial-Temporal Graph Transformer for Next Point-of-Interest Recommendation", Machine Learning and Knowledge Discovery in Databases: European Conference, (ECML-PKDD), 2023 (submitted)

\end{査読付}

\begin{公開}{1}

\bibitem{hanai-arim-mdx}
"ARIM-mdxデータシステム",
\url{https://lcnet.t.u-tokyo.ac.jp/data_system/},



\end{公開}

\begin{特許}{1}

\bibitem{hanai-iot}
華井雅俊、河村光晶、石川亮、鈴村豊太郎、
“IoT デバイス、データ転送システムおよびデータ転送方法”,
特願2023-156343,
東大TLOより出願済,


\end{特許}

\begin{発表}{1}


\bibitem{01早川智彦08}
望戸雄史,早川智彦,大西偉允,石川正俊:  トンネル内における照明認識による自己位置推定手法,令和5年度土木学会全国大会第78回学術講演会, CS9-24, 広島, 2023.9

\bibitem{01早川智彦09}
望戸雄史, 蛭間友香, 早川智彦, 柯毓珊, 大西偉允, 石川正俊:  照明認識を利用した高速道路のトンネル外観検査のための自己位置推定手法, 第21回ITSシンポジウム2023, 2-B-13, 富山, 2023.12

\bibitem{02末石智大06}
栃岡陽麻里,末石智大,石川正俊: ボールのバウンド位置予測に基づくダイナミックプロジェクションマッピングの開発, 第28回日本バーチャルリアリティ学会大会 (VRSJ2023), 論文集,3D2-01, 八王子, 2023.9

\bibitem{02末石智大07}
田畑智志,末石智大,宮下令央,石川正俊: 円筒位置姿勢情報の高速フィードバックを用いたダイナミックアナモルフォーシスシステムの開発, 第24回計測自動制御学会システムインテグレーション部門講演会 (SI2023), 講演会論文集, pp.1280-1285, 新潟, 2023.12


\bibitem{dmiya3}
八木 裕輝, 宮本 大輔, 大規模言語モデルを用いたソフトウェア脆弱性の深刻度の推定手法, コンピュータセキュリティシンポジウム, 2023年10月


% \bibitem{sms}
% 脇 聡志, 鈴村 豊太郎, 金刺 宏樹, 小林 秀, "強化学習によるマッチング数を最大化するジョブ推薦システム." 第37回人工知能学会全国大会 (2023),  一般社団法人 人工知能学会, (2023年6月発表予定)


% \bibitem{li2023nnkgc}
% Zihui Li and Boming Yang and Toyotaro Suzumura,\lq\lq NNKGC: Improving Knowledge Graph Completion with Node Neighborhoods", arXiv preprint, 2023.

% \end{発表}
\begin{発表}{1}  % other talks (Not peer reviewed)

\bibitem{ireneli-7}
Rui Yang, Haoran Liu, Edison Marrese-Taylor, Qingcheng Zeng, Ke Yuhe, Wanxin Li, Lechao Cheng, Qingyu Chen, James Caverlee, Yutaka Matsuo, Irene Li, \lq\lq KG-Rank: Enhancing Large Language Models for Medical QA with Knowledge Graphs and Ranking Techniques", 2024.

\bibitem{ireneli-8}
Hang Jiang, Xiajie Zhang, Robert Mahari, Daniel Kessler, Eric Ma, Tal August, Irene Li, Alex Pentland, Yoon Kim, Jad Kabbara, Deb Roy, \lq\lq Leveraging Large Language Models for Learning Complex Legal Concepts Through Storytelling", 2024.

\bibitem{ireneli-9}
Rui Yang, Boming Yang, Sixun Ouyang, Tianwei She, Aosong Feng, Yuang Jiang, Freddy Lecue, Jinghui Lu, Irene Li, \lq\lq Leveraging Large Language Models for Concept Graph Recovery and Question Answering in NLP Education", 2024.

\bibitem{ireneli-10}
Rui Yang, Qingcheng Zeng, Keen You, Yujie Qiao, Lucas Huang, Chia-Chun Hsieh, Benjamin Rosand, Jeremy Goldwasser, Amisha D Dave, Tiarnan D.L. Keenan, Emily Y Chew, Dragomir Radev, Zhiyong Lu, Hua Xu, Qingyu Chen, Irene Li,\lq\lq Ascle: A Python Natural Language Processing Toolkit for Medical Text Generation", 2024.

\bibitem{ireneli-11}
Fan Gao, Hang Jiang, Rui Yang, Qingcheng Zeng, Jinghui Lu, Moritz Blum, Dairui Liu, Tianwei She, Yuang Jiang, Irene Li, \lq\lq Evaluating Large Language Models on Wikipedia-Style Survey Generation", 2023.

\bibitem{ireneli-12}
Dairui Liu, Boming Yang, Honghui Du, Derek Greene, Aonghus Lawlor, Ruihai Dong, Irene Li, \lq\lq RecPrompt: A Prompt Tuning Framework for News Recommendation Using Large Language Models", 2023.




\end{発表}

\begin{特記}{1}


\end{特記}

\begin{報道}{1}



\bibitem{ireneli-13}
(Interview) Gemma Conroy, \textit{How ChatGPT and other AI tools could disrupt scientific publishing} \footnote{https://www.nature.com/articles/d41586-023-03144-w}, \textbf{Nature, Featured News}, 10 October 2023 

\end{報道}


%最後に全員分の成果をマージする。

\end{document}

