\subsection{サイバーセキュリティとデータサイエンスの融合領域に関する研究(宮本 大輔)}

サイバーセキュリティにおいて,複雑に変化するサイバー脅威の傾向を大量のデータから解析し,兆候を予測することは非常に重要である.
このため、サイバーセキュリティにとってデータサイエンスは非常に重要であり、これらの領域の密接な連携に行っている.

今年度はIoT機器向けのマルウェアの特徴に注目した分析を行い,マルウェアの挙動解析を行うシステムを開発・評価をし,この成果を論文~\cite{dmiya2}としてまとめた.また,セキュリティの脆弱性の深刻度について,脆弱性情報から予測する研究に自然言語解析技法を用いる研究を行い,この成果の発表~\cite{dmiya3}を行った.

さらに以前より継続して,財務ビッグデータの解析をテーマとした共同研究を行っている.本研究では探索的なデータ解析手法を採用しているため,インタラクティブな可視化や集約を用いてデータについての歪分布の性質を有している知見を得て解析を進め,この成果を論文~\cite{dmiya1}としてまとめた.



