\subsection{第一原理計算とデータ科学・機械学習による物質科学研究(河村 光晶)}

物質を構成する電子や原子に対して、基本法則となる(相対論的)量子力学や統計物理学に基づく理論的研究と、様々な環境(温度、圧力等)やプローブ(電子線、可視光、X線、中性子線、物理量測定等)における実験的研究の両者を協調的に行う事で、既存の現象の理解や新たなる物質探索をより効果的に進めることができる。
実験との比較を行うにあたり、物性物理学の理論を多様な組成$\cdot$構造を持つ現実の物質に適用するためには計算機によるシミュレーションが不可欠となり、スパコンやmdxのような高速$\cdot$大規模な並列計算資源およびデータ格納環境が利用される。
我々はそのような大規模計算機における物質科学シミュレーションの手法やプログラムの開発、およびそれを実際の物質$\cdot$現象に適用する研究を行っている。

二硫化モリブデン($\textrm{MoS}_2$)は化学的$\cdot$機械的安定性や高い電子易動度により、電子デバイスや触媒として用いられている。特に脱硫触媒としての性能において重要となるのが大気中での表面の硫黄欠陥の安定性やその周囲の構造変化である。
大気圧中での物質表面での特定の元素の周りの状況(化学的環境)を測定することは容易ではないが、間接的にそれを知ることができる方法の一つがX線光電子分光(XPS)である。この方法では、内核電子の束縛エネルギーが元素によって大きく異なることを利用して、元素選択的な信号を得ることができる。XPSを$\textrm{MoS}_2$に適用した結果では、硫黄(S)、モリブデン(Mo)ともに表面での束縛エネルギーの低下が観測された~\cite{kawamura_mos2}。これは両元素とも周囲の電子の量が増えていることを示唆している。またいくつかの表面構造モデルでの第一原理シミュレーションによれば、S原子が欠損したモデルにおいて計算で得られた束縛エネルギーが実験地と定量的に一致しており、欠損したアニオンSから両元素に電子が供給されるという機構を提案した。

極低温で金属の電気抵抗が消失する超伝導は、量子力学的性質が巨視的なスケールで発現する特異な現象であり、新たな超伝導物質の探索や発現機構の解明のための研究が行われている。
そのようなものの一つとして、既知の超伝導体$\textrm{La}_2\textrm{IRu}_2$のルテニウム(Ru)をオスミウム(Os)で置き換えた新規物質$\textrm{La}_2\textrm{IOs}_2$が合成され、この置換により超伝導転移温度($T_c$)が4.8 Kから12 Kへと大幅に変化することが観測された~\cite{kawamura_la2ios2}。RuとOsは同じ属の元素であり、非常に似た性質を持つため単体での$T_c$はほぼ同じであるが、この化合物ではLaとの軌道混成のために電子状態が大きく異なることを第一原理計算により明らかにした。またこれらの物質では磁場に対して超伝導が頑強であることも興味深い。
第一原理計算からの$T_c$予測による新規物質探索にも注目が集まっており、そのための手法開発や、現時点での理論の精度の検証もおこなっている~\cite{kawamura_yitp}。
超伝導に限らずあらたな熱電材料・構造材料の理論的探索として、$MAX$相の網羅的構造安定性解析を行った~\cite{kawamura_max}。
$MAX$相とは、d電子の少ないスカンジウム、チタン等の遷移金属元素($M$)、d電子の多い金属元素やアルミニウム等の典型元素($A$)と炭素または窒素($X$)からなる層状化合物の総称であり、構成元素の組み合わせや組成比($M_{n+1}AX_n$)により多様な物質が提案できる。
我々は密度半関数理論に基づく第一原理構造最適化と、格子振動解析を行い未だ合成されていないいくつかの$MAX$相とその長周期構造の提案を行い、フェルミ面の形状からその長周期構造の発現機構を議論した。

このように密度汎関数理論に基づく第一原理計算による記述が効果的な系がある一方で、そのような手法では捉えきれない研究対象が存在する。
強相関電子系とよばれる物質群では、非従来型超伝導や量子スピン液体といった全く新しい物理現象が発現することがあるが、それらの機構を解明するためのシミュレーションには厳密対角化や量子モンテカルロ法といったより高度な計算手法が必要となる。
第一原理計算の結果を前処理として、単純なモデル化を経て高度な解析を行うためのぽうろグラムパッケージの開発を行っており、
厳密対角化を行う$\mathcal{H}\Phi$~\cite{kawamura_hphi,kawamura_hpci3}と多変数変分量子モンテカルロ法をおこなうmVMC~\cite{kawamura_hpci2}がある。
これらのプログラムは東京大学物性研究所のソフトウェア高度化プロジェクトのなかで開発が行われた~\cite{kawamura_hpci1, kawamura_kotai}。
