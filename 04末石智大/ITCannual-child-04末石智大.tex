\subsection{ダイナミックビジョンシステムの研究開発(末石智大)}

%  高速画像処理および高速光学系制御を用いた、動的検査技術とヒューマンインターフェースに関するダイナミックビジョンシステムの研究を実施した。

% 動的検査技術は、実世界の動的かつ複雑な現象を適応的にデータ化し、意味のある形で活用する技術となるものであり、昨年度に引き続き本年度もマイクロサッカードと呼ばれる眼球微振動などを対象として実施した。
% 被験者を拘束する負荷や時間効率の観点から、動的状態への検査技術の発展の期待は大きいと考えられる。頭部固定を必要とせずリラックスした状態の人間のマイクロサッカード検出に向けて、回転ミラーや液体可変焦点レンズなどの光学素子を高速に制御しつつ高解像度合焦画像計測を達成することで、ダイナミックビジョンシステムにおける動的検査への基礎技術の研鑽を進めている。
% 継続して開発を進めているマイクロサッカード計測システムの改良に加え、定量的評価のための動的眼球模型の両眼化を含む開発・改良、照明制御を伴う視線方向の高精度計測戦略、マイクロサッカード検知アルゴリズムを含む統合システム開発などの成果を実現した。

% 本年度は眼球運動以外にも、卓球ボールの回転運動をリアルタイムに計測するシステムや、液体可変焦点レンズを含むカメラ系の効率的な構成手法などにも取り組んだ。
% 特に球技スポーツであるテニスのライン判定を目的とした高速ビジョンと落下位置予測技術の融合システムの初期検討も実施し、本内容で国内学会において優秀講演賞を受賞した。
% ヒューマンインターフェースに関しては、ベクター図形を描画可能なレーザー投影システムに不可視のバイナリ符号情報を埋め込んだ、自転車競技などへの活用が期待される、高速自己位置推定システムを開発し論文誌に掲載された。
% 光学系・照明系の適切な同期・校正手法を新たに開発することで、円に限定されない非対称なベクター図形を用いた情報提示と高速トラッキングの両立を可能とし、スポーツにおける複雑な身体運動のデータ化や即時フィードバックによる効率的な運動学習が期待される。

% 本年度は総じて、ヒトの眼球やボールの跳躍位置など瞬間的な動作に着目した高速センシングに関する新たな計測制御技術を、特に実世界応用へと繋がる成果として創出した。
