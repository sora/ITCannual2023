%\subsection{革新的情報環境構築に向けた能動計測型高速トラッキングシステムの研究開発(早川智彦)}

%  2022年度は主に高速画像処理を用いた高速トラッキングシステムの開発を実施し、主に1.白線認識を用いた高速道路における高速撮像システムの研究、2.高速画角切り替えによる画角拡張システムの開発を実施した。全体を通した研究成果として、2件の雑誌論文(査読付)、1件の解説論文、3件の査読付き国際学会予稿と1件の雑誌以外の査読なし国内学会予稿に採択された。

% 1.白線認識を用いた高速道路における高速撮像システムの研究
% 高速道路のトンネルにおける自己位置推定手法として、GPSでは電波を捕捉できないため機能せず、加速度センサでは十分な精度が得られないため、白線認識を利用した自己位置推定手法を開発した。この手法では、白線を画像で撮像することで、その見え方から自己位置を推定することが可能であり、自己位置を元に、トンネル天井面の撮像角度を任意の角度へと補償しながら撮像し続けることが可能となる。これにより、運転手の技量に頼ることなく、狙った位置を撮像し続けることができ、点検の効率向上に寄与する技術といえる。この成果を国際論文誌に投稿し、採択に至った。また、インフラ点検の関連技術をまとめ、関連技術を解説論文にも投稿することで、技術の社会実装が円滑に進むよう努めた。

% 2.高速画角切り替えによる画角拡張システムの開発
% 画角と解像度はトレードオフの関係にあるが、画角をミラーによって高速に切り替えることで、仮想的に画角の拡張を実現する技術を開発した。これにより、例えば高速道路の点検といった応用において、高解像度な画像を少ないカメラ台数で複数回に渡り撮像する状況において、その撮像回数を大幅に減少させることが可能となる。この成果を国際論文誌に投稿し、採択に至った。

\subsection{アクティブパーセプションとその応用展開(早川智彦)}

2023年度は主に1.高速画像処理技術による点検の高度化、2.光学的能動マーカーによるトラッキング技術の開発、3.人間の視線情報のトラッキングによる情報提示手法の開発を実施した。全体を通した研究成果として、1件の雑誌論文(査読付)、5件の雑誌以外の査読付き論文、2件の招待講演、2件の査読なしの発表を行った。

 1.高速画像処理技術による点検の高度化
インフラ点検において正確に点検対象の位置を捕捉するため、車両の自己位置の把握は必須となる。そこで、本研究ではトンネルに存在する照明装置の位置を認識することで、車両から追加の照明やレーザー照射を利用すること無く、正確に車両の自己位置を求める手法を開発し、実際に高速道路上で試すに至った。結果として、狙った位置にある特定のひび割れを時速100kmで走行しながら高分解撮像することができた。本内容を元に、論文投稿1件や国際学会発表2件、国内学会発表2件の成果を得られた。

 2.光学的能動マーカーによるトラッキング技術の開発
マーカーを用いずに動的対象をトラッキングするため、レーザー加温と紫外線照射による蓄光によって、高速にマーカーを生じさせる手法を開発し、国際学会にて発表を行った。どちらの手法もトラッキングに対して有用であることを明らかにしたため、今後それぞれの手法に適した状況にあわせて使い分けられる予定である。本成果で招待講演1件と国際学会の発表2件を行った。

 3.人間の視線情報のトラッキングによる情報提示手法の開発
人の視線情報にあわせてフレームレートを可変とすることで、人の知覚情報量を変えること無く計算機の計算量を低減させる手法を開発した。実験によって原理の検証を行い、国際学会にて発表した。
