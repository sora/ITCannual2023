\subsection{革新的情報環境構築に向けた能動計測型高速トラッキングシステムの研究開発(早川智彦)}

%  2022年度は主に高速画像処理を用いた高速トラッキングシステムの開発を実施し、主に1.白線認識を用いた高速道路における高速撮像システムの研究、2.高速画角切り替えによる画角拡張システムの開発を実施した。全体を通した研究成果として、2件の雑誌論文(査読付)、1件の解説論文、3件の査読付き国際学会予稿と1件の雑誌以外の査読なし国内学会予稿に採択された。

% 1.白線認識を用いた高速道路における高速撮像システムの研究
% 高速道路のトンネルにおける自己位置推定手法として、GPSでは電波を捕捉できないため機能せず、加速度センサでは十分な精度が得られないため、白線認識を利用した自己位置推定手法を開発した。この手法では、白線を画像で撮像することで、その見え方から自己位置を推定することが可能であり、自己位置を元に、トンネル天井面の撮像角度を任意の角度へと補償しながら撮像し続けることが可能となる。これにより、運転手の技量に頼ることなく、狙った位置を撮像し続けることができ、点検の効率向上に寄与する技術といえる。この成果を国際論文誌に投稿し、採択に至った。また、インフラ点検の関連技術をまとめ、関連技術を解説論文にも投稿することで、技術の社会実装が円滑に進むよう努めた。

% 2.高速画角切り替えによる画角拡張システムの開発
% 画角と解像度はトレードオフの関係にあるが、画角をミラーによって高速に切り替えることで、仮想的に画角の拡張を実現する技術を開発した。これにより、例えば高速道路の点検といった応用において、高解像度な画像を少ないカメラ台数で複数回に渡り撮像する状況において、その撮像回数を大幅に減少させることが可能となる。この成果を国際論文誌に投稿し、採択に至った。
