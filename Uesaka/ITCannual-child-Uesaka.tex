\subsection{生態音響システムを用いた生物調査(上坂 怜生)}

% 本節では、2022年度の上坂怜生の研究活動について報告する。本年度は生態音響システムによる高線量地域の鳥類の調査を行った。科学技術振興機構の創発的研究支援事業の一部として行っている本研究では、福島県浪江町の帰宅困難地域内での生態系の経年的変化を音響的手法で調査することを目的としている。当研究室では帰宅困難地域内に設置されたマイクにより過去複数年に渡って調査地の環境音を収集しており、生態学的な価値を持つデータが蓄積され続けている。本年度はこの生態音響データの解析に着手し始めた。実際に福島県浪江町の帰宅困難地域へ赴き、マイクの設置状況等を確認するといった定期的なメンテナンスも行った。本システムによって蓄積されている音響データには特にカラスや鶯といった鳥類の音声が多量に含まれているため、音声の特徴を抽出することでこれら鳥類の出現頻度を検証できるよう解析を進めている。

% また本年度は、生態音響による調査手法を国内のみならず発展途上国の生態系調査へと応用する可能性を鑑み、ネパールの自然保護機関であるNational Trust for Nature Conservation(NTNC)との議論を進めた。福島県の帰宅困難地域という社会的インフラ(電気やインターネット接続)のなかった場所に新たに作り上げた音響システムは、インフラ設備が整っていない発展途上国の広大な野外環境においても応用可能と考えられる。しかし、実現のためには現地の住民や政府機関を含めたステークホルダーの説得や現地研究者の情報技術リテラシー教育など、様々な問題をひとつずつ解決する必要があるため、入念な事前調査が必要である。本年度はネパールのチトワン国立公園を実際に訪れることで、どのような動物がどのような危機にさらされているか事前調査を行った。また、生態音響機材を実際にネパール国内の研究者とともに設置しながら議論を進めることで、生態音響的手法が保全活動に大きく貢献できることを確認した。ネパール国内での調査を福島県浪江町で行っている研究と並行して進めることで、生態音響システムによる生物調査のさらなる発展を促すことができると考えられる。
 