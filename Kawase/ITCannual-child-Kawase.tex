\subsection{野生動物ワイヤレスセンサネットワークと時空間行動分析に関する研究(川瀬 純也)}
%  本研究室では、野生動物装着型ワイヤレスセンサーネットワーク(以下:野生動物 WSN)機構による自然環境等でのデータ収集手法の開発と、それによって得られるデータの解析手法についての研究を行っている。人間が容易には侵入できないエリアでの継続的なデータ収集機構として種々の社会問題の解決に寄与することを目指している。
 
% 2022年度は、コロナ禍において中断を余儀なくされていた家畜等の動物による各種実験・データの収集を再開するべく、その環境の再構築に重点を置いて研究活動を進めてきた。動物による実験は専門家の指導の下、実験内容や対象となる動物に合わせた各種設備・環境を整えて行う必要があるため、その環境の構築と整備は慎重かつ丁寧な確認を繰り返して行う必要がある。2022年夏より、北海道のある乳牛農家に実験の協力を依頼すべく、地元の家畜用機器を扱う業者と協力しながら検討・議論を進めている。

%  また、野生動物を対象に含む時空間行動の分析の手法として、配列アライメント手法を用いた類型化手法に注目し、研究を進めている。配列アライメント手法は、もともとはバイオインフォマティクスで用いられる手法であり、編集距離という概念と動的計画法によって、文字列間の類似度を定量的に算出し、類型化する手法である。筆者は博士論文時からこの手法の時空間行動分析への応用手法の検討を行ってきた。野生動物の時空間行動は、その意図や目的などを明示的に把握することができず、GNSSデータなどの移動データから推測するしかない。多種多様で大量な移動データを分析する上で、定量的な類型化手法は重要となる。現在活用可能な移動データを用いて手法の検討を進めており、2023年中に順次報告を行う予定である。