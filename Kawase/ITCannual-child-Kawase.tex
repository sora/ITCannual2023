\subsection{野生動物ワイヤレスセンサネットワークと時空間行動分析に関する研究(川瀬 純也)}
本研究室では、野生動物装着型ワイヤレスセンサーネットワーク機構による自然環境下でのデータ収集手法の開発と、それによって得られるデータの解析手法についての研究を行っている。人間が容易には侵入できないエリアでの継続的なデータ収集機構として種々の社会問題の解決に寄与することを目指している。

2023年度は、数年来コロナ禍において中断を余儀なくされていた家畜等の動物による各種実験・データの収集に重点を置いて活動を開始した。協力関係にある企業に依頼し、北海道の広い放牧地で自由に移動し活動する乳牛のGPSデータ等の収集を行った。2箇所の放牧場において、それぞれ6か月・4か月に渡る継続的な放牧牛のGPSデータを収集することができた。これらのGPSデータを用いて、遅延耐性ネットワーク (DTN)技術を用いたワイヤレスセンサーネットワークのシミュレーションを行い、被覆面積の効率的な最大化手法などを検討していく。
 
野生動物を対象に含む時空間行動分析の手法として、配列アライメント手法を用いた類型化手法に着目し、研究を進めている。配列アライメント手法は、もともとはバイオインフォマティクスで用いられる手法であり、編集距離の概念と動的計画法によって、文字列間の類似度を定量的に算出し、類型化する手法である。野生動物の時空間行動は、その意図や目的などを明示的に把握することができず、GPSデータなどの移動データから推測するしかない。そのGPSデータも自然環境下では測位精度が低く、野生動物装着型デバイスのバッテリー持続期間の問題からも、測位の間隔やタイミングが整ったデータを収集することは困難である。多種多様で、大量かつ欠落部分を含む移動データを分析する上で、これらの問題点を考慮した定量的な類型化手法は非常に重要となる。
 
そこで、2023年度に収集した家畜のGPSデータを用いて、これらの類型化手法について検討を進めている。特に、DTN技術を用いた野生動物装着型ワイヤレスセンサーネットワークにおいては、異なる群れの間を行き来したり、積極的に他の個体と接触したりする個体の存在が重要となる。そのような個体が、広くデータを伝播させる役割を担うことができると考えられるからである。類型化手法においては、「一緒に行動する群れ」を特定するだけでなく、「群れの間を行き来するような特徴的な少数派」を効率的に見つけ出すことを目的のひとつとしている。これらの研究成果をまとめ、2024年度中には積極的に報告を進めていく。
