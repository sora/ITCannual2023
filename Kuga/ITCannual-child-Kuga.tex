\subsection{データセンタハードウェアへのソフトウェア脆弱試験の適応(空閑 洋平)}

% 現在のデータセンタ環境では、機械学習やニューラルネットワークの学習,推論を高速化する専用アクセラレータが広く使用されるようになった。専用アクセラレータを用いた計算環境は、既存のCPUを中心に構成されていたソフトウェア環境に比べて、プロセッサやデバイスドライバ、デバイス間通信が専用に設計され、CPUをバイパスしてデバイス間で直接データ通信されるため、デバイスのデータ通信の把握や可視化が困難なブラックボックス化が進んでいる。今後、専用アクセラレータを中心とした次世代のデータセンタ環境では、CPUをバイパスするデバイス間通信が増加することで、セキュリティ監視や脆弱性試験、管理手法、データ通信内容の可視化手法といった、普段CPU環境で実施している運用課題が顕在化すると考えられる。

% 今年度は、CPUからデータセンタハードウェアを直接操作するために設計した独自Remote DMA機能を用いた研究を実施した\cite{ykuga39987672,ykuga41835070,ykuga37056192}。
% 特に、今年度はソフトウェアでプログラム可能なメモリデバイスを提案し、ソフトウェアでアクセラレータやストレージのデータ通信内容の観測と書き換えが可能なことを確認できたため、来年度はアクセラレータに対する脆弱性試験を予定している。
% また、昨年から引き続き、CPUをバイパスするNIC型ネットワークルータアーキテクチャの検討を実施し、今年度は経路表のみをCPUで処理するハイブリットアプローチを提案している\cite{ykuga36919054}。
% mdxの機能高度化に関する研究については、データ転送機能の高性能化手法の検討と、mdx上でのkubernetes基盤構築に関する機能開発について招待論文で報告した\cite{ykuga41835081,ykuga41534619}。
% その他の活動としては、東京大学のZoomデータを用いて、広域ネットワーク品質解析の手法を提案した\cite{ykuga40356877}。
