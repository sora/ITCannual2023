% \subsection{小型高速三次元スキャナの開発および可変光学系による技術拡張に関する研究(田畑智志)}

%  本年度は主に小型高速三次元スキャナの開発を昨年度に引き続き実施するとともに、計測システムの可変焦点化による計測技術の向上方法の検討と、形状計測の高精度高解像化やダイナミックプロジェクションマッピング技術の広範囲化に関する実験を実施した。

% 高速三次元スキャン技術は、運動する物体の形状取得やロボットにおける外界認識・インタラクションなど、高速性・リアルタイム性が要求される応用において重要である。昨年度に引き続き、小型ユニットを用いた1,000fpsでの高速小型三次元スキャナの開発を行い、国内学会での口頭発表や招待論文、報道による周知を行った。
% リアルタイムのモデル生成と運動補正を組みあわせることで1,000fpsの高速性を維持したまま、1辺10cmの立方体に収まるサイズで片手に持ってスキャン可能な計測システムを実現している。また、この技術をさらに発展させるため、さらなる統合アプローチの検討を進めている。

% さらに、三次元形状計測について、計測距離範囲・精度を向上させるため高速可変焦点レンズの導入を進め予備実験を実施した。また、位相シフト法における計測精度の向上や高解像化に際して課題となる点を精査し、それぞれの課題を解決するためのアルゴリズムの開発を進めている。

% また、ダイナミックプロジェクションマッピングに関しては、対象の三次元的な情報を計測・利用するシステムに対する回転ミラーを用いたトラッキングアプローチの拡張を進めており、実際の応用を通じた実験を行っている。


